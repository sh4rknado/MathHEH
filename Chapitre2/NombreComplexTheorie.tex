\newpage

\section{Nombres Complexes Théorie}
\vspace{5mm} %5mm vertical space

\subsection{Conversion polaire - cartésienne}
\vspace{5mm} %5mm vertical space

Définition du module: \\

le module noté $|Z|$ est la longueur du segment (rayon). Elle peut être mesurée  grâce à la formule de pythagore ($\sqrt{a²+b²}$). \\

Démonstration : \\

$|Z| = \rho cos(\theta)+ \rho sin(\theta) *i$ \\
$|Z| = \sqrt(\rho^{2} cos(\theta)^{2}+ \rho^{2} sin(\theta)^{2})$ \\
$|Z| = \sqrt(\rho^{2} cos(\theta)^{2}+ sin(\theta))*i $ \\
$|Z| = \sqrt(\rho^{2}) $ \\
$|Z| = \rho $ \\

$\rho$ est le module et $\theta$ est l'argument \\
$Z= P(cos(\theta) + sin(\theta)*i )$ ou $Z= P(cis(\theta))$\\

\vspace{5mm} %5mm vertical space
Représentation Géographique \\

\begin{tikzpicture}
  \tkzInit[xmin=-5,xmax=5,ymin=-5,ymax=5]
  \tkzGrid[sub,color=gray, subxstep=.5,subystep=.5]
  \tkzAxeXY[very thick]
  \tkzGrid
\end{tikzpicture}

\newpage

\subsection{Conversion Cartésienne - Polaire}
\vspace{5mm} %5mm vertical space

$\rho$ = $\sqrt(x²+y²)$ \\

Démonstration Géométriquement $\theta$ \\

Nous pouvons voir que $\theta$ est modifié en fonction de X et de Y que si nous dessinons un cercle, nous pouvons voir que le segment Y est une tangeante au cercle de rayon X. \\

\begin{tikzpicture}
  \tkzInit[xmin=-5,xmax=5,ymin=-5,ymax=5]
  \tkzGrid[sub,color=gray, subxstep=.5,subystep=.5]
  \tkzAxeXY[very thick]
  \tkzGrid
\end{tikzpicture}

$X=\rho * cos(\theta)$ $ $ $ Y=\rho * sin(\theta)$

\vspace{5mm} %5mm vertical space
Démonstration Algébriquement $\theta$ \\

$\frac{Y}{X}$ = $\frac{\rho * sin(\theta)}{\rho * cos(\theta)}$ \\
$\frac{Y}{X}$ = $\frac{sin(\theta)}{cos(\theta)}$ \\
$\frac{Y}{X}$ = tg($\theta)$ \\

\vspace{5mm} %5mm vertical space
Conclusion : \\

$\theta = arctg(\frac{Y}{X})$
