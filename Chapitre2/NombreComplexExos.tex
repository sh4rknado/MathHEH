\newpage

\chapter{Nombres Complexes Exercices}
\vspace{5mm} %5mm vertical space
\section{Enoncés}

\vspace{5mm} %5mm vertical space
\textbf{1) Résoudre les équations suivantes}

\begin{itemize}
\item {a. x²+1=0}
\item {b. 3x²+7=0}
\item {c. $\frac{x²}{2} -x=-2$}
\item {d. -x²-3x=3}
\item {e. x³+7x²+9x+63=0}
\item {f. $x^{4}$ +15x²=16}
\end{itemize}

\vspace{3mm} %5mm vertical space
\textbf{2) Trouver le conjugués de }

\begin{itemize}
\item {a. -11-8i}
\item {b. -0.3333i + 1}
\item {c. $cos(\omega t) + sin(\omega t)i$}
\end{itemize}

\vspace{3mm} %5mm vertical space
\textbf{3) Identifier \R $  $ \I}

\begin{itemize}
\item {a. 0}
\item {b. -6+i}
\item {c. i²}
\item {d. $\frac{1+i}{2}$}
\end{itemize}


\vspace{3mm} %5mm vertical space
\textbf{4) Exprimer sous forme a+bi}

\begin{itemize}
\item {a. (4-8i)-(3+2i)}
\item {b. $\frac{3}{3+2i} + \frac{1}{5-i}$}
\item {c. $(7-2i)(5+6i)$}
\item {d. $\frac{4}{(3+i)³}$}
\item {e. $\frac{5+3i}{(2+2i)} $}
\item {f. $\frac{3+6i}{(3-4i)} $}
\item {g. $(\frac{1+i}{2-i})^{2}$ + $\frac{3+6i}{3-4i}$}
\item {h. $\frac{2+5i}{1-i}$ + $\frac{2-5i}{1+i}$}
\item {i. Nombre de modules 2 et d'argument $\frac{\pi}{3}$}
\item {j. Nombre de modules 3 et d'argument $\frac{-\pi}{8}$}
\end{itemize}

\vspace{3mm} %5mm vertical space
\textbf{5) Exprimer sous forme Polaire}

\begin{itemize}
\item {a. 3-$\sqrt(3i)$}
\item {b. -1+1i}
\end{itemize}

\vspace{3mm} %5mm vertical space
\textbf{6) Exprimer sous forme cartésienne}

\begin{itemize}
\item {a. 4cos(45) + sin(45)i}
\item {b. $5cis(\frac{\pi}{3})$}
\end{itemize}

\vspace{3mm} %5mm vertical space
\textbf{7) Trouver la solution de}

\begin{itemize}
\item {a. 4cis(45°)+5cis($\frac{\pi}{3}$)}
\item {b. 4cis(45°)*5cis($\frac{\pi}{3}$)}
\end{itemize}

\vspace{3mm} %5mm vertical space
\textbf{8) changer de formes}

\begin{itemize}
\item {a. $6*cis(30^{\circ})$ en forme exp}
\item {b. $e^{e^{1 + \frac{\pi}{2}*i}} $}
\item {c. $1 + \sqrt{3i}$ en forme exp}
\end{itemize}

\vspace{3mm} %5mm vertical space
\textbf{8) donner la valeure de}

\begin{itemize}
\item {a. module de $3e^{\frac{\pi}{4}*i}$ }
\item {b. argument de $3e^{\frac{\pi}{4}*i}$}
\item {c. Re($2e^{-\pi*i}$) }
\item {c. $I(2e^{-\pi*i}$) }
\end{itemize}


\newpage
\section{Résoudre les équations suivantes}
\vspace{3mm} %5mm vertical space

\textbf{A. x²+1 = 0} \\

x²+1-1=0-1 \\
x²= -1 \\
x=$\sqrt{-1}$ \\
S = x=i \\

\vspace{5mm} %5mm vertical space
\textbf{B. 3x²+7 = 0} \\

3x²+7-7=0-7 \\

$\frac{3x²}{3}$= $\frac{-7}{3}$ \\

x² = $\frac{-7}{3}$ \\
$\sqrt{x²}$ = $\sqrt{\frac{7}{3} *-1}$ \\
$\sqrt{x²}$ = $\sqrt{\frac{7}{3}}$ $\sqrt{-1}$ \\
S = $\sqrt{x²}$ = $\sqrt{\frac{7}{3}}$ $\sqrt{-1}$ \\

\vspace{5mm} %5mm vertical space
\textbf{C. $\frac{x²}{2}$ -x = -2} \\

$\frac{x²}{2}$ - $\frac{x}{1}$ = - $\frac{2}{1}$\\

$\frac{x²}{2}$ - $\frac{2x}{2}$ = - $\frac{4}{2}$\\

$\frac{x²}{2}$ - $\frac{2x}{2}$ = - $\frac{4}{2}$\\

x² - 2x = -4 \\
x² - 2x + 4 = (-4)+4 \\
x² - 2x + 4 = 0 \\

$\frac{-2+-\sqrt{(-2)²-4*1*4}}{2*1}$ \\
$\frac{-2+-\sqrt{4 - 16}}{2}$ \\
$\frac{-2+-\sqrt{-12}}{2}$ \\
$\frac{-2+-\sqrt{4*(-3)}}{2}$ \\
$\frac{-2+-\sqrt{(2)²*(-3)}}{2}$ \\
S= -1 +- 1 $\sqrt{-3}$ \\


\newpage
\textbf{D. -x²-3x = 3} \\

-x²-3x -3 = 3-3 \\
-x²-3x -3 = 0 \\

$\frac{-3+-\sqrt{(3)²-4*1*3}}{2*1}$ \\

$\frac{-3+-\sqrt{9-12}}{2}$ \\

$\frac{-3+-\sqrt{-3}}{2}$ \\

$\frac{-3+-\sqrt{3 * (-1)}}{2}$ \\

$\frac{-3+-\sqrt{3} * \sqrt{-1}}{2}$ \\

$\frac{-3+-\sqrt{3i}}{2}$ = -$\frac{3}{2} +- \sqrt{\frac{3}{2}i}$ \\

\vspace{10mm} %5mm vertical space
\textbf{E. x³+7x²+9x+63 = 0} \\

x²+(x+7)+9(x+7)=0 \\

(x+7)*(x²+9)=0 \\

Poser les CE pour que (x+7) ou (x²+9) vaut 0 \\

Résoudre pour (x+7)=0 \\

x=-7 \\

(x²+9)=0 \\

x²=-9 \\

$\sqrt{x²} =\sqrt{-3²}$ \\

$\sqrt{x²}=\sqrt{3²*(-1)}$ \\

$x=3\sqrt{-1}$ \\

$x=3i$ \\

S= X vaut -7;3i \\

\newpage

\textbf{F. $x^{4}$ +15x² = 16} \\

$x^{4}$+15x²-16=0 \\

Poser t = x² \\

t²+15t-16=0 \\

t*(t+16)-(t+16) = 0 \\

(t+16)*(t-1)=0 \\


CE : Les Possibilités que la solution vaut 0 quand :

\begin{itemize}
\item {t+16=0}
\item {t-1=0}
\end{itemize}

(t+16) = 0 \\

t = (-16) \\

\vspace{5mm}

Restituer t=x² \\

x²=-16 \\

$x=\sqrt{-16}$ \\

$x=\sqrt{16 * (-1)}$ \\

$x=\sqrt{4² * (-1)}$ \\

$x=4\sqrt{-1}$ \\

x=4i \\

t-1=0 \\
t=1 \\

\vspace{5mm}

Restituer t=x² \\

x²=1 \\
$x=\sqrt{1}$ \\
x=1 \\


S = 1; 4i \\

\newpage

\section{Trouver le conjugués}

\begin{itemize}
\item {a. -11-8i = -11+8i}
\item {b. -0.3333i + 1 = 1+0.3333i}
\item {c. $cos(\omega t) + sin(\omega t)i$ = $cos(\omega t) - sin(\omega t)i$ }
\end{itemize}


\section{Identifier \R $  $ \I}

\begin{itemize}
\item {a. 0 : \R=0 \I=0 }
\item {b. -6+i : \R=(-6) \I=1}
\item {c. i²: \R=(-1) \I=0}
\item {d. $\frac{1+i}{2}$ : \R=($\frac{1}{2}$) \I=($\frac{1}{2}$)}
\end{itemize}


\section{Exprimer sous forme a+bi}

\begin{itemize}
\item {a. (4-8i)-(3+2i) : 1-10i}
\item {b. $\frac{3}{3+2i} + \frac{1}{5-i}$ : $\frac{23-11i}{26}$}
\item {c. $(7-2i)(5+6i)$ : 47+32i}
\item {d. $\frac{4}{(3+i)³}$ : $\frac{9-13i}{125}$}
\item {e. $\frac{5+3i}{(2+2i)}$ : 2-$\frac{1}{2}i$}
\end{itemize}
\vspace{8mm} %5mm vertical space1

f. $\frac{3+6i}{(3-4i)} $ \\

\textbf{Etape 1 : Binomes conjugués} \\

$\frac{3+6i}{(3-4i)}$ * $\frac{3+4i}{(3+4i)}$ = $\frac{9+12i+18i+24i²}{9-16i²}$ \\

\textbf{Etape 2 : Par définition $i^{2} = (-1)$} \\

$\frac{9+30i+(24*(-1)}{9-16*(-1)}$  = $\frac{9+30i+(-24)}{9-(-16)}$ \\

$\frac{9+(-24)+30i}{9+16}$  = $\frac{-15+30i}{25}$ \\

\textbf{Etape 3 : Factoriser} \\

$\frac{5*(-3+6i)}{5*5}$ = $\frac{(-3+6i)}{5}$ \\

\textbf{Etape 4 : Exprimer sous la forme a+bi} \\

$\frac{-3}{5}$ + $\frac{6i}{5}$ \\

\newpage

g. $(\frac{1+i}{2-i})^{2}$ + $\frac{3+6i}{3-4i}$ \\

\textbf{Etape 1 : utilisation de $(a+b)^{2}$ = $a^{2}+2ab+b^{2}$} \\

$(\frac{1}{5}$ + $\frac{3}{5}*i)$ - $\frac{3}{5}$ + $\frac{6}{5}*i$ \\

\textbf{Etape 2 : Mise au même dénominateur} \\

$(\frac{1}{25}$ + $\frac{6}{25}*i)$ - $\frac{9}{25}*(-1)$ - $\frac{3}{5}$ + $\frac{6}{5}i$ \\

$(\frac{-23}{25}$ + $\frac{6}{25}*i)$ + $\frac{6}{5}i$ \\

$\frac{-23}{25}$ + $\frac{36}{25}i$ \\

\vspace{5mm} %5mm vertical space1

h. $\frac{2+5i}{1-i}$ + $\frac{2-5i}{1+i}$ \\

\textbf{Etape 1: Réduire au même dénominateur (1-i)*(1+i)} \\

$\frac{(1+i) * (2+5i) + (1-i) * (2-5i)}{ (1-i) * (1+i) }$ \\

\textbf{Etape 2 : Distributivités} \\

$\frac{2+2i+5i+5i^{2} + 2-2i-5i+5i^{2} }{1-i+i-i^{2}}$ \\

$\frac{4 + 10i^{2} }{1-i^{2}}$ \\

\textbf{Etape 3 : Par définition $i^{2} = -1$} \\

$\frac{4 + (10*(-1)) }{1-(1*(-1))}$ \\

$\frac{4-10}{2}$ = $-\frac{6}{2} = -3$ \\

\vspace{5mm} %5mm vertical space1

\textbf{i. Nombre de modules 2 et d'argument $\frac{\pi}{3}$} \\

$|Z| = 2*cis(\frac{\pi}{3})$ \\

$X= \rho * cos(\theta) => X = 2*cos(\frac{\pi}{3})$  \\
$Y = \rho * sin(\theta) => Y = 2*sin(\frac{\pi}{3})$ \\

$X = 2*\frac{1}{2} = 1$ \\
$Y = 2\sqrt{\frac{3}{2}}$ \\

\textbf{Exprimer sous la forme a+bi} \\

$S = 1 + \sqrt{\frac{6}{2}} i = 1 + \sqrt{3} i$ \\

\newpage

\textbf{j. Nombre de modules 3 et d'argument $\frac{-\pi}{8}$} \\

DEMANDER EXPLICATION


\newpage
\vspace{3mm} %5mm vertical space
\section{Exprimer sous forme polaire}
\vspace{3mm} %5mm vertical space

a. 3-$\sqrt{3i}$ \\

\textbf{Calcul de l'argument} \\
\vspace{3mm} %5mm vertical space

$\theta = arctg(\frac{-\sqrt{3}} {3})$ \\
$\theta = -30^{\circ}$ \\
$\theta = -30^{\circ} + 360^{\circ}$ \\
$\theta = 330^{\circ}$ \\

\textbf{Calcul du module} \\
\vspace{3mm} %5mm vertical space

$\rho = \sqrt{3²+(-\sqrt{3})²}$ \\
$\rho = \sqrt{9+3}$ \\
$\rho = \sqrt{12 => (12=4*3)}$ \\
$\rho = \sqrt{2²*3}$ \\
$\rho = 2\sqrt{3}$ \\

Z=$\rho * cos(\theta)*sin(\theta)*i => \rho * cis(\theta)$ \\

Z=$ 2\sqrt{3} * cis(330)^{\circ} $ \\

\vspace{5mm} %5mm vertical space

b.  -1+1i \\

\textbf{Calcul de l'argument} \\
\vspace{3mm} %5mm vertical space

$\theta = arctg(-\frac{1} {1})$ \\
$\theta = -45^{\circ}$ \\
$\theta = -45^{\circ} + 360^{\circ}$ \\
$\theta = 315^{\circ}$ \\

\textbf{Calcul du module} \\
\vspace{3mm} %5mm vertical space

$\rho = \sqrt{-1²+1²}$ \\
$\rho = \sqrt{2}$ \\

Z=$\rho * cos(\theta)*sin(\theta)*i => \rho * cis(\theta)$ \\

Z=$ \sqrt{2} * cis(315^{\circ}) $ \\

\newpage

\vspace{3mm} %5mm vertical space
\section{Exprimer sous forme cartésienne}
\vspace{3mm} %5mm vertical space

a. $4cos(45^{\circ}) + sin(45^{\circ}) *i$ \\

\textbf{Formules} \\
$\rho = 4 * cis(45^{\circ})$\\
$\theta = arctg(\frac{Y}{X})$ \\
$|Z| = a+bi $\\

\vspace{3mm} %5mm vertical space
$\frac{Y}{X} = tg(45^{\circ})$\\
$\frac{Y}{X} = 1$\\

$\rho = \sqrt{x²+y²} = 4$\\
$\rho = \sqrt{(x²+y²)}² = 4²$\\
$\rho = x²+y² = 16$\\

Notes : $\frac{Y}{X} = 1 = \frac{1}{1}$ $ $ donc Y=X \\

$\rho = 2x² = 16$ $ $ ou $ 2y² = 16$ \\
$\rho = x² = \frac{16}{2}$ \\
$\rho = x² = 8$ \\
$\rho = \sqrt{x²} = \sqrt{8 = (2*4)}$ \\
$\rho = x = \sqrt{(2*2²)}$ \\
$\rho = x = 2\sqrt{2}$ $ $ et $ $ $ y=2\sqrt{2} $\\

x=y donc $x = 2\sqrt{2}$ $ $ et $ $ $ y=2\sqrt{2i} $ \\

\textbf{Conclusion}\\

$ S= 4 * cis(45^{\circ}) = 2\sqrt{2} + 2\sqrt{2i}$

\newpage

b. $5 * cis(\frac{\pi}{3})$ \\

\textbf{Formules} \\

$\rho = 5 $\\
$\theta = arctg(\frac{Y}{X})$ \\
$|Z| = a+bi $\\

\vspace{3mm} %5mm vertical space
$\theta = tg(\frac{\pi}{3})$\\
$\theta = \sqrt(3)$\\

$x= \rho * cos(\sqrt{3}) => cos(\sqrt{3}) = \frac{1}{2}$ \\
$y= \rho * sin(\sqrt{3}) => sin(\sqrt{3}) = \frac{\sqrt{3}}{2}$ \\

$x= 5*\frac{1}{2} = \frac{5}{2}$ \\
$y= 5*\frac{\sqrt{3}}{2}$ \\

\textbf{Conclusion} \\

Z= a+bi \\

$ S= Z=\frac{5}{2} + 5*\frac{\sqrt{3i}} {2}$

\newpage

\vspace{3mm} %5mm vertical space
\section{Trouver la solution}
\vspace{3mm} %5mm vertical space

$a. 4*cis(45°) + 5*cis(\frac{\pi}{3})$\\

\textbf{Calcul du module} \\
\vspace{3mm} %5mm vertical space

$\rho = \sqrt{\rho_1^{2}+\rho_2^{2} + 2 * \rho_1 * \rho_2 * cos(\theta_1-\theta_2))}$ \\

$\rho = \sqrt{4^{2}+5^{2} + 2*4*5 * cos(45^{\circ}-60^{\circ}))}$ \\

$\rho = \sqrt{16+25 + 40 * cos(-15^{\circ}))}$ \\

$\rho = \sqrt{41 + 40 * cos(-15^{\circ}))}$ \\

$\rho = \sqrt{81 * 0.965}$ \\

$\rho = \sqrt{79.637}$ \\

$\rho = 8.9239 $ \\

\textbf{Calcul de l'argument} \\
\vspace{3mm} %5mm vertical space

$\theta = arctg(\frac{Y}{X})$ \\

$\theta = arctg(\frac{\rho_1 * sin(\theta_1) + \rho_2 * sin(\theta_2)} {\rho_1 * cos(\theta_1) + \rho_2 * cos(\theta_2)})$ \\

$\theta = arctg(\frac{4 * sin(45^{\circ}) + 5 * sin(60^{\circ})} {4 * cos(45^{\circ}) + 5 * cos(60^{\circ})})$ \\

$\theta = arctg(\frac{4\frac{\sqrt{2}} {2} + 5\frac{\sqrt{3}} {2}} {4\frac{\sqrt{2}} {2} + 5\frac{1} {2})})$ \\

$\theta = arctg(1,343)$ \\

$\theta = 53,338^{\circ}$ \\

$ S = 4*cis(45°) + 5*cis(\frac{\pi}{3}) = 8.9239 * cis(53.338^{\circ})$

\newpage

$b. 4*cis(45°) * 5*cis(\frac{\pi}{3})$ \\

\textbf{Calcul du module} \\
\vspace{3mm} %5mm vertical space

$\rho = \sqrt{\rho_1*\rho_2 (cos(45^{\circ} + \theta_2)+ i* sin(45^{\circ}+ \theta_2)) }$ \\

$\rho = \sqrt{4*5 (cos(45^{\circ} + 60^{\circ})+ i* sin(45^{\circ}+ 60^{\circ})) }$ \\

$\rho = \sqrt{20  (\frac{\sqrt{2}}{2} + \frac{1}{2}) + \frac{\sqrt{2}} {2} + \frac{\sqrt{3}} {2} }$\\

$\rho = \sqrt{24,1421 + 1,5731 }$\\

$\rho = \sqrt{25,7152}$\\

$\rho = 5,07$\\

\textbf{Calcul de l'argument} \\
\vspace{3mm} %5mm vertical space

$\theta = arctg(\frac{Y}{X})$\\

$\theta = arctg(\frac{\rho_1 * sin(\theta_1) + \rho_2 * sin(\theta_2)} {\rho_1 * cos(\theta_1) + \rho_2 * cos(\theta_2)})$ \\

$\theta = arctg(\frac{4 * sin(45^{\circ}) + 5 * sin(60^{\circ})} {4 * cos(45^{\circ}) + 5 * cos(60^{\circ})})$ \\

$\theta = arctg(\frac{4\frac{\sqrt{2}} {2} + 5\frac{\sqrt{3}} {2}} {4\frac{\sqrt{2}} {2} + 5\frac{1} {2})})$ \\

$\theta = arctg(1,343)$ \\

$\theta = 53,338^{\circ}$ \\

$ S = 4*cis(45°) + 5*cis(\frac{\pi}{3}) = 8.9239 * cis(53.338^{\circ})$ \\

\newpage
\section{changement de forme (exp)}
\vspace{5mm}

\textbf{a. $6*cis(30^{\circ})$ en forme exp} \\

$\rho*cis(\theta) = \rho * e^{\theta i}$\\

$6 cis(30^{\circ}) = 6e^{30^{\circ}i} = 6e^{\frac{\pi}{6}i}$ \\

$S = 6e^{\frac{\pi}{6}i}$ \\

\vspace{6mm}
\textbf{b. $e^{1 + \frac{\pi}{2}i}$} \\

\textbf{Mettre sous la forme a + bi} \\

$e^{1} + e^{\frac{\pi}{2}i}$ \\

\textbf{Calculer $e*cis(\frac{\pi}{2})$} \\

$e * ( cos(\frac{\pi}{2}) + i*sin(\frac{\pi}{2}) ) $ \\

$e * ( 0 + 1i ) $ \\

$S = e * i $ \\

\vspace{6mm}
\textbf{c. $1 + \sqrt{3i}$ en forme exp} \\

\vspace{3mm}
\textbf{Etape 1: Trouver $\rho$ (calcul du module)} \\

$\rho = \sqrt{x^{2} + y^{2}} = \sqrt{1^{2} + \sqrt{3}^{2}}$ \\

$\rho = \sqrt{1 + 3} = \sqrt{2^{2}}$ \\

$\rho = 2$ \\

\vspace{3mm}
\textbf{Etape 2: Trouver $\theta$ (calcul de l'argument)} \\

$tg(\theta) = arctg(\frac{1}{\sqrt{3}})$ \\

$tg(\theta) = (\frac{1}{\sqrt{3}} * \frac{\sqrt{3}}{\sqrt{3}}) = \frac{1\sqrt{3}}{\sqrt{3}^{2}} = \frac{\sqrt{3}}{3}$ ou $\frac{\pi}{6}$ \\

\vspace{3mm}
\textbf{Etape 3: Ecriture sous le format exponentielle} \\

$\rho*cis(\theta) = \rho * e^{i\theta} = 2e^{\frac{\pi}{6}i}$ \\

\newpage
\section{Recherche valeures (exp)}

\textbf{a. module de $3e^{\frac{\pi}{4}*i}$} \\

\textbf{b. argument de $3e^{\frac{\pi}{4}*i}$} \\

\textbf{c. Re($2e^{-\pi*i}$)} \\

\textbf{d. $I(2e^{-\pi*i}$)} \\
