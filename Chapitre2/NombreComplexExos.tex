\newpage

\section{Nombres Complexes Exercices}
\vspace{5mm} %5mm vertical space
\subsection{Exercices : Enoncés}

\vspace{5mm} %5mm vertical space
1) Résoudre les équations suivantes:

\begin{itemize}
\item {a. x²+1=0}
\item {b. 3x²+7=0}
\item {c. $\frac{x²}{2} -x=-2$}
\item {d. -x²-3x=3}
\item {e. x³+7x²+9x+63=0}
\item {f. $x^{4}$ +15x²=16}
\end{itemize}

\vspace{3mm} %5mm vertical space
2) Trouver le conjugués de :

\begin{itemize}
\item {a. -11-8i}
\item {b. -0.3333i + 1}
\item {c. $cos(\omega t) + sin(\omega t)i$}
\end{itemize}

\vspace{3mm} %5mm vertical space
3) Identifier \R $  $ \I

\begin{itemize}
\item {a. 0}
\item {b. -6+i}
\item {c. i²}
\item {d. $\frac{1+i}{2}$}
\end{itemize}


\vspace{3mm} %5mm vertical space
4) Exprimer sous forme a+bi :

\begin{itemize}
\item {a. (4-8i)-(3+2i)}
\item {b. $\frac{3}{3+2i} + \frac{1}{5-i}$}
\item {c. $(7-2i)(5+6i)$}
\item {d. $\frac{4}{(3+i)³}$}
\item {e. $\frac{5+3i}{(2+2i)} $}
\end{itemize}

\vspace{3mm} %5mm vertical space
5) Exprimer sous forme Polaire :

\begin{itemize}
\item {a. 3-$\sqrt(3i)$}
\item {b. -1+1i}
\end{itemize}

\vspace{3mm} %5mm vertical space
6) Exprimer sous forme cartésienne :

\begin{itemize}
\item {a. 4cos(45) + sin(45)i}
\item {b. $5cis(\frac{\pi}{3})$}
\end{itemize}

\vspace{3mm} %5mm vertical space
7) Trouver la solution de :

\begin{itemize}
\item {a. 4cis(45°)+5cis($\frac{\pi}{3}$)}
\item {b. 4cis(45°)*5cis($\frac{\pi}{3}$)}
\end{itemize}


\vspace{3mm} %5mm vertical space
\subsection{Résoudre les équations suivantes:}
\vspace{3mm} %5mm vertical space

A. x²+1=0 \\

x²+1-1=0-1 \\
x²= -1 \\
x=$\sqrt{-1}$ \\
S = x=i \\

\vspace{3mm} %5mm vertical space
B. 3x²+7=0 \\

3x²+7-7=0-7 \\

$\frac{3x²}{3}$= $\frac{-7}{3}$ \\

x² = $\frac{-7}{3}$ \\
$\sqrt{x²}$ = $\sqrt{\frac{7}{3} *-1}$ \\
$\sqrt{x²}$ = $\sqrt{\frac{7}{3}}$ $\sqrt{-1}$ \\
S = $\sqrt{x²}$ = $\sqrt{\frac{7}{3}}$ $\sqrt{-1}$ \\


\vspace{3mm} %5mm vertical space

C. $\frac{x²}{2}$ -x = -2 \\

$\frac{x²}{2}$ - $\frac{x}{1}$ = - $\frac{2}{1}$\\

$\frac{x²}{2}$ - $\frac{2x}{2}$ = - $\frac{4}{2}$\\

$\frac{x²}{2}$ - $\frac{2x}{2}$ = - $\frac{4}{2}$\\

x² - 2x = -4 \\
x² - 2x + 4 = (-4)+4 \\
x² - 2x + 4 = 0 \\

$\frac{-2+-\sqrt{(-2)²-4*1*4}}{2*1}$ \\

$\frac{-2+-\sqrt{4 - 16}}{2}$ \\
$\frac{-2+-\sqrt{-12}}{2}$ \\
$\frac{-2+-\sqrt{4*(-3)}}{2}$ \\
$\frac{-2+-\sqrt{(2)²*(-3)}}{2}$ \\
S= -1 +- 1 $\sqrt{-3}$ \\


\vspace{3mm} %5mm vertical space

D. -x²-3x=3 \\

-x²-3x -3=3-3 \\
-x²-3x -3=0 \\

$\frac{-3+-\sqrt{(3)²-4*1*3}}{2*1}$ \\

$\frac{-3+-\sqrt{9-12}}{2}$ \\
$\frac{-3+-\sqrt{-3}}{2}$ \\
$\frac{-3+-\sqrt{3 * (-1)}}{2}$ \\
$\frac{-3+-\sqrt{3} * \sqrt{-1}}{2}$ \\
$\frac{-3+-\sqrt{3i}}{2}$ \\
S = -$\frac{3}{2} +- \sqrt{\frac{3}{2}i}$ \\

\vspace{3mm} %5mm vertical space

E. x³+7x²+9x+63=0 \\

x²+(x+7)+9(x+7)=0 \\
(x+7)*(x²+9)=0 \\

Poser les CE pour que (x+7) ou (x²+9) vaut 0 \\

Résoudre pour (x+7)=0 \\
x=-7 \\
(x²+9)=0 \\
x²=-9 \\
$\sqrt{x²} =\sqrt{-3²}$ \\
$\sqrt{x²}=\sqrt{3²*(-1)}$ \\
$x=3\sqrt{-1}$ \\
$x=3i$ \\

S= X vaut -7;3i \\


\vspace{10mm} %5mm vertical space
F. $x^{4}$ +15x²=16 \\

$x^{4}$+15x²-16=0 \\
Poser t = x² \\
t²+15t-16=0 \\
t*(t+16)-(t+16)=0 \\
(t+16)-(t-1)=0 \\

CE : Les Possibilités que la solution vaut 0 quand : \\

(t+16)=0 \\
t=-16 \\
Restituer t=x² \\
x²=-16 \\
$x=\sqrt{-16}$ \\
$x=\sqrt{16 * (-1)}$ \\
$x=\sqrt{4² * (-1)}$ \\
$x=4\sqrt{-1}$ \\
x=4i \\


t-1=0 \\
t=1 \\
Restituer t=x² \\
x²=1 \\
$x=\sqrt{1}$ \\
x=1 \\

S= 1; 4i \\

\newpage

\vspace{3mm} %5mm vertical space
\subsection{Trouver le conjugués:}
\vspace{3mm} %5mm vertical space

\begin{itemize}
\item {a. -11-8i = -11+8i}
\item {b. -0.3333i + 1 = 1+0.3333i}
\item {c. $cos(\omega t) + sin(\omega t)i$ = $cos(\omega t) - sin(\omega t)i$ }
\end{itemize}


\vspace{3mm} %5mm vertical space
\subsection{Identifier \R $  $ \I}
\vspace{3mm} %5mm vertical space

\begin{itemize}
\item {a. 0 : \R=0 \I=0 }
\item {b. -6+i : \R=(-6) \I=1}
\item {c. i²: \R=(-1) \I=0}
\item {d. $\frac{1+i}{2}$ : \R=($\frac{1}{2}$) \I=($\frac{1}{2}$)}
\end{itemize}


\vspace{3mm} %5mm vertical space
\subsection{Exprimer sous forme a+bi}
\vspace{3mm} %5mm vertical space

\begin{itemize}
\item {a. (4-8i)-(3+2i) : 1-10i}
\item {b. $\frac{3}{3+2i} + \frac{1}{5-i}$ : $\frac{23-11i}{26}$}
\item {c. $(7-2i)(5+6i)$ : 47+32i}
\item {d. $\frac{4}{(3+i)³}$ : $\frac{9-13i}{125}$}
\item {e. $\frac{5+3i}{(2+2i)}$ : 2-$\frac{1}{2}i$}
\end{itemize}


\vspace{3mm} %5mm vertical space
\subsection{Exprimer sous forme polaire}
\vspace{3mm} %5mm vertical space

a. 3-$\sqrt{3i}$ \\

Calcul de l'arguments \\
\vspace{3mm} %5mm vertical space

$\theta = arctg(\frac{-\sqrt{3}} {3})$ \\
$\theta = -30^{\circ}$ \\
$\theta = -30^{\circ} + 360^{\circ}$ \\
$\theta = 330^{\circ}$ \\

Calcul du module \\
\vspace{3mm} %5mm vertical space

$\rho = \sqrt{3²+(-\sqrt{3})²}$ \\
$\rho = \sqrt{9+3}$ \\
$\rho = \sqrt{12 => (12=4*3)}$ \\
$\rho = \sqrt{2²*3}$ \\
$\rho = 2\sqrt{3}$ \\

Z=$\rho * cos(\theta)*sin(\theta)*i => \rho * cis(\theta)$ \\
Z=$ 2\sqrt{3} * cis(330)^{\circ} $ \\

\newpage

b.  -1+1i \\

Calcul de l'arguments \\
\vspace{3mm} %5mm vertical space

$\theta = arctg(-\frac{1} {1})$ \\
$\theta = -45^{\circ}$ \\
$\theta = -45^{\circ} + 360^{\circ}$ \\
$\theta = 315^{\circ}$ \\

Calcul du module \\
\vspace{3mm} %5mm vertical space

$\rho = \sqrt{-1²+1²}$ \\
$\rho = \sqrt{2}$ \\

Z=$\rho * cos(\theta)*sin(\theta)*i => \rho * cis(\theta)$ \\
Z=$ \sqrt{2} * cis(315^{\circ}) $ \\

\vspace{3mm} %5mm vertical space
\subsection{Exprimer sous forme cartésienne}
\vspace{3mm} %5mm vertical space

a. $4cos(45^{\circ}) + sin(45^{\circ}) *i$ \\

Formules \\

$\rho = 4 * cis(45^{\circ})$\\
$\theta = arctg(\frac{Y}{X})$ \\
$|Z| = a+bi $\\

\vspace{3mm} %5mm vertical space
$\frac{Y}{X} = tg(45^{\circ})$\\
$\frac{Y}{X} = 1$\\

$\rho = \sqrt{x²+y²} = 4$\\
$\rho = \sqrt{(x²+y²)}² = 4²$\\
$\rho = x²+y² = 16$\\
Notes : $\frac{Y}{X} = 1 = \frac{1}{1}$ $ $ donc Y=X \\

$\rho = 2x² = 16$ $ $ ou $ 2y² = 16$ \\
$\rho = x² = \frac{16}{2}$ \\
$\rho = x² = 8$ \\
$\rho = \sqrt{x²} = \sqrt{8 = (2*4)}$ \\
$\rho = x = \sqrt{(2*2²)}$ \\
$\rho = x = 2\sqrt{2}$ $ $ et $ $ $ y=2\sqrt{2} $\\

x=y donc $x = 2\sqrt{2}$ $ $ et $ $ $ y=2\sqrt{2i} $ \\

Conclusion :\\

$ S= 4 * cis(45^{\circ}) = 2\sqrt{2} + 2\sqrt{2i}$

\newpage

b. $5 * cis(\frac{\pi}{3})$ \\

Formules \\
\vspace{3mm} %5mm vertical space

$\rho = 5 $\\
$\theta = arctg(\frac{Y}{X})$ \\
$|Z| = a+bi $\\

\vspace{3mm} %5mm vertical space
$\theta = tg(\frac{\pi}{3})$\\
$\theta = \sqrt(3)$\\

$x= \rho * cos(\sqrt{3}) => cos(\sqrt{3}) = \frac{1}{2}$ \\
$y= \rho * sin(\sqrt{3}) => sin(\sqrt{3}) = \frac{\sqrt{3}}{2}$ \\


$x= 5*\frac{1}{2} = \frac{5}{2}$ \\
$y= 5*\frac{\sqrt{3}}{2}$ \\

\vspace{2mm} %5mm vertical space
Conclusion :\\
\vspace{2mm} %5mm vertical space

Z= a+bi \\
$ S= Z=\frac{5}{2} + 5*\frac{\sqrt{3i}} {2}$

\vspace{3mm} %5mm vertical space
\subsection{Trouver la solution}
\vspace{3mm} %5mm vertical space

$a. 4*cis(45°) + 5*cis(\frac{\pi}{3})$\\

$\rho = \sqrt{\rho_1^{2}+\rho_2^{2} + 2 * \rho_1 * \rho_2 * cos(\theta_1-\theta_2))}$ \\
$\rho = \sqrt{4^{2}+5^{2} + 2*4*5 * cos(45^{\circ}-60^{\circ}))}$ \\
$\rho = \sqrt{16+25 + 40 * cos(-15^{\circ}))}$ \\
$\rho = \sqrt{41 + 40 * cos(-15^{\circ}))}$ \\
$\rho = \sqrt{81 * 0.965}$ \\
$\rho = \sqrt{79.637}$ \\
$\rho = 8.9239 $ \\

$\theta = arctg(\frac{Y}{X})$\\
$\theta = arctg(\frac{\rho_1 * sin(\theta_1) + \rho_2 * sin(\theta_2)} {\rho_1 * cos(\theta_1) + \rho_2 * cos(\theta_2)})$ \\

$\theta = arctg(\frac{4 * sin(45^{\circ}) + 5 * sin(60^{\circ})} {4 * cos(45^{\circ}) + 5 * cos(60^{\circ})})$ \\

$\theta = arctg(\frac{4\frac{\sqrt{2}} {2} + 5\frac{\sqrt{3}} {2}} {4\frac{\sqrt{2}} {2} + 5\frac{1} {2})})$ \\

$\theta = arctg(1,343)$ \\

$\theta = 53,338^{\circ}$ \\

$ S = 4*cis(45°) + 5*cis(\frac{\pi}{3}) = 8.9239 * cis(53.338^{\circ})$ \\

\newpage
$b. 4*cis(45°) * 5*cis(\frac{\pi}{3})$ \\

$\rho = \sqrt{\rho_1*\rho_2 (cos(45^{\circ} + \theta_2)+ i* sin(45^{\circ}+ \theta_2)) }$ \\

$\rho = \sqrt{4*5 (cos(45^{\circ} + 60^{\circ})+ i* sin(45^{\circ}+ 60^{\circ})) }$ \\

$\rho = \sqrt{20  (\frac{\sqrt{2}}{2} + \frac{1}{2}) + \frac{\sqrt{2}} {2} + \frac{\sqrt{3}} {2} }$\\

$\rho = \sqrt{24,1421 + 1,5731 }$\\

$\rho = \sqrt{25,7152}$\\

$\rho = 5,07$\\

$\theta = arctg(\frac{Y}{X})$\\
$\theta = arctg(\frac{\rho_1 * sin(\theta_1) + \rho_2 * sin(\theta_2)} {\rho_1 * cos(\theta_1) + \rho_2 * cos(\theta_2)})$ \\

$\theta = arctg(\frac{4 * sin(45^{\circ}) + 5 * sin(60^{\circ})} {4 * cos(45^{\circ}) + 5 * cos(60^{\circ})})$ \\

$\theta = arctg(\frac{4\frac{\sqrt{2}} {2} + 5\frac{\sqrt{3}} {2}} {4\frac{\sqrt{2}} {2} + 5\frac{1} {2})})$ \\

$\theta = arctg(1,343)$ \\

$\theta = 53,338^{\circ}$ \\

$ S = 4*cis(45°) + 5*cis(\frac{\pi}{3}) = 8.9239 * cis(53.338^{\circ})$ \\
