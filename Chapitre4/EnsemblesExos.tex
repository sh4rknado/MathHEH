\newpage

\section{Théorie naïve des ensembles Exercices}
\vspace{5mm} %5mm vertical space

\subsection{Produit Cartésiens}
\vspace{3mm} %5mm vertical space

\begin{itemize}
\item {a. Reflexivité : pour tout ensemble A (A$\in$B)}
\item {b. Anti-Symétrique : (A$\in$B) et (B$\in$A) $=>$ A=B}
\item {c. Transitivité : (A$\in$B) et (B$\in$C) $=>$ (A$\in$C)}
\end{itemize}

\subsection{Exercices Examen}
\vspace{3mm} %5mm vertical space

Soit N est l’esemble des naturels sauf 0 \\
$R=\{(a,b), a ∈ N, b ∈ N | $ a est un multiple de b \} \\

\vspace{3mm} %5mm vertical space
cochez ce qui est vrai concernant R : \\

\begin{itemize}[label=$\square$]
\item {a. R est transitif}
\item {b. Aucune réponse}
\item {c. R est réflexif}
\item {d. R est anti-symètrique}
\item {e. R est symètrique}
\end{itemize}

\vspace{5mm} %5mm vertical space

Test de la symétrie \\

A=2 B=4 \\
A est multipe de B : 2*4 VRAI \\
B est multipe de A : 4*2 FAUX \\

il faut que A et B soit vrai pour qu'il soit symétrique \\

\vspace{3mm} %5mm vertical space
R est réflexif car $a ∈ N, b ∈ N$


\newpage

Soit N est l’esemble des naturels sauf 0 \\
$R=\{(a,b), a ∈ N, b ∈ N |$ a est $>$ b \} \\

\vspace{2mm} %5mm vertical space

cochez ce qui est vrai concernant R: \\

\begin{itemize}[label=$\square$]
\item {a. R est transitif}
\item {b. Aucune réponse}
\item {c. R est réflexif}
\item {d. R est anti-symètrique}
\item {e. R est symètrique}
\end{itemize}

\vspace{4mm} %5mm vertical space

Test de la symétrie \\

A est plus grand que B $=>$ VRAI \\
B est plus grand que A $=>$ FAUX \\
il faut que A et B soit vrai \\

R n'est Symètrique pas car A=1 B=2 \\
R est anti-symétrique $a est > b$ \\


R est réflexif car $a ∈ N, b ∈ N$ \\
R n'est pas transitif car a est $>$ b  et a $\neq$ b \\


\vspace{10mm} %5mm vertical space


Soit N est l’esemble des naturels sauf 0 \\
$R=\{(a,b), a ∈ N, b ∈ N |$ b est divisible a \} \\

cochez ce qui est vrai concernant R : \\

\begin{itemize}[label=$\square$]
\item {a. R est transitif}
\item {b. Aucune réponse}
\item {c. R est réflexif}
\item {d. R est anti-symètrique}
\item {e. R est symètrique}
\end{itemize}

Test de la symétrie \\

A est divisible par B $=>$ VRAI \\
B est divisible par A $=>$ VRAI \\
il faut que A et B soit vrai \\

R est symétrique car b est divisible a B=2 A=1 \\

Test de la transitivité \\

B=2 A=2 Z=A \\
A est divisible par B $=>$ VRAI \\
B est divisible par A $=>$ VRAI \\

A=B 2=2 \\
B=A 2=2 \\
alors A=Z 2=2 \\
