\newpage

\section{Théorie naïve des ensembles Exercices}
\vspace{5mm} %5mm vertical space

\subsection{Exercices Examen}
\vspace{3mm} %5mm vertical space

A) Soit A=\{pi,2,e\} et B=\{-1, 5\} Calculer ${|A\times B|}$ \\

\vspace{4mm} %5mm vertical space

  1) Calculer ${A\times B}$ \\

  A*B = \{ (pi,-1),(pi,5), (2,-1),(2,5), (e,-1),(e,5) \}

  \vspace{8mm} %5mm vertical space

    2) Calculer la cardinalité de ${|A\times B|}$ \\

      ${|A|}$ = 3  ${|B|}$ = 2 \\

      ${|A\times B|}$ = ${|A|}*{|B|}$ = 2*3 = 6 \\

      S = la cardinalité est le nombre de sous-ensembles (6)\\

  \vspace{8mm} %5mm vertical space

  B) Soit P $|$ A U B $|$ A =\{3,4,5\} B=\{1,2,3\} \\

  \vspace{5mm} %5mm vertical space

  1) Union des 2 ensembles \\

  P(A) = \{\{\}, \{3\}, \{4\}, \{5\}, \{3,4\}, \{4,5\}, \{3,5\}, \{3,4,5\}\} \\

  \vspace{4mm} %5mm vertical space
  2) Calcul de la cardinalité des ensembles \\

  $|P|$ = 8 \\
