\newpage

\section{Théorie naïve des ensembles Exercices}
\vspace{5mm} %5mm vertical space

\subsection{Enoncé d'exercices}
\vspace{3mm} %5mm vertical space

\begin{itemize}
\item {a. Soit A=\{pi,2,e\} et B=\{-1, 5\} Calculer ${|A\times B|}$}
\item {b. Soit P $|$ A U B $|$ A =\{3,4,5\} B=\{1,2,3\}}
\item {c. Soit A=\{$\pi$, 2, e\} et B=\{-1,5\} Calculer $|A U B|$}
\end{itemize}

\subsection{Résolution}
\vspace{4mm} %5mm vertical space

A) Calculer ${A\times B}$ \\

A*B = \{ (pi,-1),(pi,5), (2,-1),(2,5), (e,-1),(e,5) \}

\vspace{8mm} %5mm vertical space

2) Calculer la cardinalité de ${|A\times B|}$ \\

1) Union des 2 ensembles a 1 membre \\

P(A) = \{\{\}, \{$\pi$\}, \{2\}, \{e\}, \{-1\}, \{5\}\} \\

Total des ensembles = 6 \\

2) Union des 2 ensembles a 2 membres \\

P(A) = \{\{$\pi$,2\}, \{2,e\}, \{e,-1\}, \{-1,5\}, \{5,$\pi$\}\} \\

Total des ensembles = 5 \\

3) Union des 2 ensembles a 3 membres \\

P(A) = \{\{$\pi$,2,e\}, \{2,e,-1\}, \{e,-1,5\}, \{-1,5,$\pi$\}, \{5,$\pi$,2\}\} \\

Total des ensembles = 5 \\

3) Union des 2 ensembles a 4 membres \\

P(A) = \{\{$\pi$,2,e,-1\}, \{2,e,-1,5\}, \{e,-1,5,$\pi$\}, \{-1,5,$\pi$,2\}, \{5,$\pi$,2,e\}\} \\

Total des ensembles = 5 \\

4) Union des 2 ensembles a 5 membres \\

P(A) = \{\{$\pi$,2,e,-1,5\}\} \\

Total des ensembles = 1 \\

7) Calculer la cardinalité de P(A): \\

La sommes de la cardinalité des sous ensembles = 6 + (3*5) +1 = 22 \\

P $|$ A U B $|$ = 22 \\

S=22 \\

\newpage

B) Soit P $|$ A U B $|$ A =\{3,4,5\} B=\{1,2,3\} \\
\vspace{5mm} %5mm vertical space

1) Union des 2 ensembles a 1 membre \\

P(A) = \{\{\}, \{3\}, \{4\}, \{5\}, \{1\}, \{2\}, \{3\}\} \\

Total des ensembles =7 \\

2) Union des 2 ensembles a 2 membres \\

P(A) = \{\{3,4\}, \{4,5\}, \{5,1\}, \{1,2\}, \{2,3\}, \{3,3\}\} \\

Total des ensembles =6 \\

3) Union des 2 ensembles a 3 membres \\

P(A) = \{\{3,4,5\}, \{4,5,1\}, \{5,1,2\}, \{1,2,3\}, \{2,3,3\}, \{3,3,4\}\} \\

Total des ensembles =6 \\

3) Union des 2 ensembles a 4 membres \\

P(A) = \{\{3,4,5,1\}, \{4,5,1,2\}, \{5,1,2,3\}, \{1,2,3,3\}, \{2,3,3,4\}, \{3,3,4,5\}\} \\

Total des ensembles =6 \\

4) Union des 2 ensembles a 5 membres \\

P(A) = \{\{3,4,5,1,2\}, \{4,5,1,2,3\}, \{5,1,2,3,3\}, \{1,2,3,3,4\}, \{2,3,3,4,5\}, \{3,3,4,5,1\}\} \\

Total des ensembles =6 \\

6) Union des 2 ensembles a 6 membres \\

P(A) = \{\{3,4,5,1,2,3\}\} \\

Total des ensembles =1 \\

7) Calculer la cardinalité de P(A):\\

La sommes de la cardinalité des sous ensembles = 7 + (4*6) +1 = 32\\

P $|$ A U B $|$ = 32 \\

S=32 \\

\newpage

c) Soit A=\{$\pi$, 2, e\} et B=\{-1,5\} Calculer $|A U B|$

1) Union des 2 ensembles a 1 membre \\

P(A) = \{\{\}, \{$\pi$\}, \{2\}, \{e\}, \{-1\}, \{5\}\} \\

Total des ensembles = 6 \\

2) Union des 2 ensembles a 2 membres \\

P(A) = \{\{$\pi$,2\}, \{2,e\}, \{e,-1\}, \{-1,5\}, \{5,$\pi$\}\} \\

Total des ensembles = 5 \\

3) Union des 2 ensembles a 3 membres \\

P(A) = \{\{$\pi$,2,e\}, \{2,e,-1\}, \{e,-1,5\}, \{-1,5,$\pi$\}, \{5,$\pi$,2\}\} \\

Total des ensembles = 5 \\

3) Union des 2 ensembles a 4 membres \\

P(A) = \{\{$\pi$,2,e,-1\}, \{2,e,-1,5\}, \{e,-1,5,$\pi$\}, \{-1,5,$\pi$,2\}, \{5,$\pi$,2,e\}\} \\

Total des ensembles = 5 \\

4) Union des 2 ensembles a 5 membres \\

P(A) = \{\{$\pi$,2,e,-1,5\}\} \\

Total des ensembles = 1 \\

7) Calculer la cardinalité de P(A):\\

La sommes de la cardinalité des sous ensembles = 6 + (3*5) +1 = 22\\

P $|$ A U B $|$ = 22 \\

S=22 \\
