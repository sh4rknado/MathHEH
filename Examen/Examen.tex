
\newpage
\chapter{Exemple d'examen}
\vspace{10mm} %5mm vertical space
\section{Q1 : Calcul du déterminant de la matrice}

\vspace{10mm} %5mm vertical space

$
A =
\begin{pmatrix}
  1 & 5 & 6 & 7 \\
  0 & 2^0-1 & 1-2^32^{-3} & 8 \\
  9 & 9,5 & -9,5 & b \\
  4 & 8 & 16 & 32 \\
\end{pmatrix}
$

\vspace{10mm} %5mm vertical space

A) Simplification de la matrice \\

\vspace{2mm} %5mm vertical space
$2^{0}-1 = 1-1 = 0$
\vspace{2mm} %5mm vertical space

$1-2^{3} 2^{-3} = 1-2^{3-3} = 1-2^{0} = 1-1 = 0$
\vspace{4mm} %5mm vertical space

$
\begin{pmatrix}
  1 & 5 & 6 & 7 \\
  0 & 0 & 0 & 8 \\
  9 & 9,5 & -9,5 & b \\
  4 & 8 & 16 & 32 \\
\end{pmatrix}
$

\vspace{10mm} %5mm vertical space

B) Swap des zeros\\

\vspace{5mm} %5mm vertical space

$
8*
\begin{pmatrix}
  1 & 5 & 6 \\
  9 & 9,5 & {-9,5} \\
  4 & 8 & 16 \\
\end{pmatrix}
$

\vspace{10mm} %5mm vertical space

C) Extraction des sous matrices \\

\vspace{5mm} %5mm vertical space
Matrices de signes \\

\vspace{5mm} %5mm vertical space

$
\begin{pmatrix}
  + & + & - \\
  - & - & + \\
  + & + & - \\
\end{pmatrix}
$


\vspace{10mm} %5mm vertical space
Extraction des matrices \\
\vspace{5mm} %5mm vertical space

$
8*(
  1*
  \begin{pmatrix}
    9,5 & -9.5 \\
    8 & 16 \\
  \end{pmatrix}
  $
  $
  -5*
  \begin{pmatrix}
    9,5 & -9.5 \\
    4 & 16 \\
  \end{pmatrix}
  $
  $
  +6*
  \begin{pmatrix}
    9 & 9.5 \\
    4 & 8 \\
  \end{pmatrix}
  $
)

\newpage

D) Calcul des déterminants 2*2


\vspace{5mm} %5mm vertical space

8*(\\

1*((9,5*16)-(8*-9,5))\\

-5*((9*16)-(4*-9,5))\\

+6*((9*8)-(4*9,5))\\

)

\vspace{10mm} %5mm vertical space

E) Simplification des calculs

\vspace{5mm} %5mm vertical space

8*(\\

  1*(152 – (-76))\\

  -5*(144 – (-38))\\

  +6*(72 – 38)\\

)

\vspace{10mm} %5mm vertical space


F) Mise en équation et résolution \\

\vspace{5mm} %5mm vertical space

8*( 228 -5*(182) + 6*(34))\\

8*( 228 - 910 + 204 )\\

8*( 228 + 204 - 910 )\\

8*( 432 - 910 )\\

8*( -478 ) = -3824\\

det(A) = -3824\\


\newpage

\section{Q2 : Calcul nombre complex}

\vspace{4mm} %5mm vertical space

Que doit valoir a pour que l’argument soit 135° quand b=-5, c=4 et d=11 \\

\vspace{4mm} %5mm vertical space

$
\frac{a+bi}{c+di}
$

\vspace{5mm} %5mm vertical space

A) ?????????????

\newpage

\vspace{10mm} %5mm vertical space

\section{Q3 : Transformer en forme conjonctive}

\vspace{4mm} %5mm vertical space

(A ∧¬ B) V $(C\implies a)$ \\

\begin{flushleft}
A) Simplifier l’implications \\
\end{flushleft}

  (A ∧¬ B) V (¬ A V (C ∧ A))\\

\begin{flushleft}
B) Utilisation du théorème De Morgan \\
\end{flushleft}

  $a+b = ¬ a * ¬ b$\\

  ¬ (A ∧¬ B) ∧ ¬ (¬ A V (C ∧ A)) \\

  S= NEG(A ET NEG(B))ETNEG(NEG(A)OU(C OU A)) \\


\vspace{10mm} %5mm vertical space

\section{Q4 : Théorie des ensembles naïfs}

\vspace{4mm} %5mm vertical space

A) Soit A=\{pi,2,e\} et B=\{-1, 5\} Calculer ${|A\times B|}$ \\

\vspace{4mm} %5mm vertical space

  1) Calculer ${A\times B}$ \\

  A*B = \{ (pi,-1),(pi,5), (2,-1),(2,5), (e,-1),(e,5) \}

  \vspace{8mm} %5mm vertical space

    2) Calculer la cardinalité de ${|A\times B|}$ \\

      ${|A|}$ = 3  ${|B|}$ = 2 \\

      ${|A\times B|}$ = ${|A|}*{|B|}$ = 2*3 = 6 \\

      S = la cardinalité est le nombre de sous-ensembles (6)\\

  \vspace{8mm} %5mm vertical space

  B) Soit P $|$ A U B $|$ A =\{3,4,5\} B=\{1,2,3\} \\

  \vspace{5mm} %5mm vertical space

  1) Union des 2 ensembles \\

  P(A) = \{\{\}, \{3\}, \{4\}, \{5\}, \{3,4\}, \{4,5\}, \{3,5\}, \{3,4,5\}\} \\

  \vspace{4mm} %5mm vertical space
  2) Calcul de la cardinalité des ensembles \\

  $|P|$ = 8 \\

\newpage
\section{Q5 : Induction forte/faibles}

\vspace{5mm} %5mm vertical space

Notez que l’induction faible est égale à l’induction forte. Néanmoins il est plus naturel de démontrer les propriétés soit avec de l’induction simple,soit avec la forte comme réalisé durant le cours. Il vous est demandé de choisir entre les deux fonction de l’énoncé.\\ \\
Soit n un nombre naturel, que faut-il pour démontrer que $10^{n-1}$ est un multiple de 9 ? \\ \\
Veuillez choisir au moins une réponse : (Cochez ce qui est vrai)\\

\begin{itemize}[label=$\square$]
\item On peut utiliser l’induction faible ou forte
\item Il faut au moins 3 cas de base
\item il faut utiliser l’induction forte
\item il faut au moins un unique cas de base
\item il faut au moins 2 cas de base
\end{itemize}

\vspace{5mm} %5mm vertical space

\section{Q6 : Nombre entiers}

\vspace{5mm} %5mm vertical space


Soient a,b et m des nombre naturels. Est-ce que

\vspace{2mm} %5mm vertical space

(a+b) mod m = ((a mod m)+(b mod m)) mod m \\

\vspace{4mm} %5mm vertical space

a) Développement de l'égalité \\

(a+b) mod m = ((a+b) mod m) mod m

(8+10) mod 2 = ((8 mod 2)+(10 mod 2)) mod 2

(18) mod 2 = (0+0) mod 2

0 = (0) mod 2

0 = 0 \\

Sélectionnez une réponse :
\begin{itemize}[label=$\square$]
\item Vrai
\item Faux
\end{itemize}


\newpage
\section{Q7 : Déterminer les complexités de l’algorithme suivant avec n la taille du tableau}

\vspace{5mm} %5mm vertical space

\begin{lstlisting}[language=Python, caption=Python algorithme]

def Apply(array,value,start=None,res=0):

	if(start is None):
            start = len(array)-1

        if(start <0):
            return res

	if(array[start] == value):
            return Apply(array,value,start-1,res+1)

return Apply(array,value,start-1,res)
\end{lstlisting}

\vspace{5mm} %5mm vertical space

cochez ce qui est vrai concernant la complexités (au moins une réponse)\\
\begin{itemize}[label=$\square$]
\item {a. $\theta(1)$}
\item {b. $o(n²)$}
\item {c. $O(log(n))$}
\item {d. $o(log(n))$}
\item {e. $\theta(log(n))$}
\item {f. $o(n)$}
\item {g. $o(1)$}
\item {h. $O(1)$}
\item {i. $o(n log(n))$}
\item {j. $O(n log(n))$}
\item {k. $\theta(n)$}
\item {l. $\theta(n²)$}
\item {m. $O(n²)$}
\item {n. $O(n)$}
\item {o. $\theta(n log(n))$}
\end{itemize}

\newpage
\section{Q8 : Ensemble Naturels}
\vspace{5mm} %5mm vertical space

Soit N est l’esemble des naturels sauf 0

  R={(a,b), a ∈ N, b ∈ N et a est un multiple de b }\\


cochez ce qui est vrai concernant R. (au moins une réponse)\\
\begin{itemize}[label=$\square$]
\item R est transitif
\item Aucune réponse
\item R est réflexif
\item R est anti-symètrique
\item R est symètrique
\end{itemize}
