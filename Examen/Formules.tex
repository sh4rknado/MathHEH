
\newpage
\chapter{Formules}

\section{Tableau Trigonométrique}
\begin{tabular}{|l|c|c|c|c|c|c|}
  \hline
  Degree & $0^{\circ}$ & $30^{\circ}$ & $45^{\circ}$  & $60^{\circ}$ & $90^{\circ}$ \\
  Radians & 0 & $\frac{\pi}{6}$ & $\frac{\pi}{4}$ & $\frac{\pi}{3}$ &  $\frac{\pi}{2}$ \\
  \hline
  sin & 0 & $\frac{1}{2}$ & $\frac{\sqrt{2}} {2}$ & $\frac{\sqrt{3}} {2}$ & 1 \\
  cos & 1 & $\frac{\sqrt{3}} {2}$ & $\frac{1}{2}$ & $\frac{\sqrt{2}} {2}$ & 0 \\
  tan & 0 & $\frac{\sqrt{3}} {3}$ & 1 & $\sqrt{3}$ & $\nexists$ \\
  cotan & $\nexists$ & $\sqrt{3}$ & 1 & $\frac{\sqrt{3}} {3}$ & 0 \\
  \hline
\end{tabular}

\vspace{4mm} %5mm vertical space
\includegraphics[scale=0.3]{circle_angles}

\newpage
\section{NB Complex : Forme Polaire vers Cartésienne}
$X= \rho * cos(\theta)$ \\
$Y= \rho * sin(\theta)$ \\
$Z= x+yi $ \\
Notes : $cis = cos(\theta) * sin(\theta) *i$ \\

\vspace{4mm} %5mm vertical space
\section{Addition de nombres complex (cartésien)}

Exemple : (a+bi) + (a+di) \\

($a_1$+$a_2$) + ($b_1$+$b_2$) *i \\

\vspace{4mm} %5mm vertical space
\section{Soustraction de nombres complex (cartésien)}

Exemple : (a+bi) - (a+di) \\

($a_1$-$a_2$) + ($b_1$-$b_2$) *i \\

\vspace{4mm} %5mm vertical space
\section{Multilication de nombres complex (cartésien)}

Exemple : (a+bi) * (a+di) \\

($a_1$*$a_2$) - ($b_1$*$b_2$) + (($a_1$ * $b_2$) + ($b_1$*$a_2$) ) *i \\

\vspace{4mm} %5mm vertical space
\section{Division de nombres complex (cartésien)}

Exemple : $\frac{(a+bi)}{(a+di)}$ \\

$\frac{(a_1 * a_2) - (b_1*b_2)} {a_{2}^{2}+b_{2}^{2} } $ + $\frac{(b_1 * a_2) - (a_1*b_2)} {a_{2}^{2}+b_{2}^{2} } $ *i \\

\newpage
\vspace{4mm} %5mm vertical space
\section{NB Complex : Forme cartésienne vers polaire}

$\rho = \sqrt{x²+y²}$ \\
$\theta = arctg(\frac{Y}{X})$ \\
$\frac{Y}{X} = tg(\theta)$ \\

\vspace{4mm} %5mm vertical space
\section{Addition de nombres complex (Polaire)}

$ Exemple : 4*cis(45^{\circ}) + 5*cis(\frac{\pi}{3})$\\

$\rho = \sqrt{\rho_1^{2}+\rho_2^{2} + 2 * \rho_1 * \rho_2 * cos(\theta_1-\theta_2))}$ \\

$\theta = arctg(\frac{\rho_1 * sin(\theta_1) + \rho_2 * sin(\theta_2)} {\rho_1 * cos(\theta_1) + \rho_2 * cos(\theta_2)})$ \\

\vspace{4mm} %5mm vertical space
\section{Soustraction de nombres complex (Polaire)}

$ Exemple : 4*cis(45^{\circ}) - 5*cis(\frac{\pi}{3})$\\

$\rho = \sqrt{\rho_1^{2}+\rho_2^{2} + 2 * \rho_1 * \rho_2 * cos(\theta_1-\theta_2))}$ \\

$\theta = arctg(\frac{\rho_1 * sin(\theta_1) + \rho_2 * sin(\theta_2)} {\rho_1 * cos(\theta_1) + \rho_2 * cos(\theta_2)})$ \\

\vspace{4mm} %5mm vertical space
\section{Multilication de nombres complex (Polaire)}

$ Exemple : 4*cis(45^{\circ}) * 5*cis(\frac{\pi}{3})$\\

c1*c2 = $\rho_1$*$\rho_2$ *(cos($\theta_1$ + $\theta_2$) + i * sin($\theta_1$ + $\theta_2$)) \\

\vspace{4mm} %5mm vertical space
\section{Division de nombres complex (Polaire)}

$ Exemple : \frac{(a+bi)}{(c+di)}$\\

$\frac{c1}{c2} = \frac{r1}{r2} * cos(\theta_1+\theta_2) + i * sin(\theta_1-\theta_2)$ \\

\vspace{8mm} %5mm vertical space

Notes : Selon l'énoncé et les préférences de chacun il est conseillé de transformer en forme polaire ou cartésien,\\
afin de pouvoir appliquer les formules ci-dessus.\\


\newpage
\section{Logique propositionnelle}

\vspace{5mm} %5mm vertical space
De Morgans: \\
a v b= ¬a * ¬b \\
a*b= ¬a + ¬b \\
(p∧q) = ¬p v ¬q  \\
(pvq) = ¬ (¬p ∧ ¬q)  \\
¬(p∧q) = (p v q)  \\
(A ∧¬ B) V (¬ A V (C ∧ A)) =  ¬(A ∧¬ B) ∧ ¬(¬ A V (C ∧ A)) \\

\vspace{5mm} %5mm vertical space
Forme disjonctive \\
(A ∧ B) V C \\
(A ET B) OU C \\

\vspace{5mm} %5mm vertical space
Forme conjonctive \\
(A V B) ∧ C\\
(A OU B) ET C \\

\vspace{5mm} %5mm vertical space
Transformation: \\
A$=>$B = ¬A v (A∧B) \\
A$<=>$B = (A$=>$B)∧(B$=>$A)  \\
(A$=>$B)∧(B$=>$A) = (¬A v (A∧B)) ∧ (¬B v (B∧A))


\section{Algorithmique symbole}

\vspace{5mm} %5mm vertical space

o = meilleur des cas \\

O = Pire des cas \\

$\theta$ = Cas moyen \\

$\Uptheta$ = Meilleur des cas, cas moyen, pire des cas \\
