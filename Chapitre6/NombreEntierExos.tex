\newpage

\chapter{Nombre Entiers Exercices}

\vspace{3mm} %5mm vertical space
\section{Exercices Modulo}
\vspace{3mm} %5mm vertical space

\begin{itemize}

\item {1. Prouver que si a ∈ N, a $>$ 0, alors 1 divise a et a divise 0. }

\item {2. Soient a, b ∈ Z, a 6 = 0, b 6 = 0, prouvez que si a$|$b et b$|$a alors a = b ou a = −b.}

\item {3. Déterminez le quotient et le reste de 111 divisé par 11; 123 par 7; 777
divisé par 21; 1434 divisé par 13 et 1025 divisé par 15.}

\item {4.Calculez 7 mod 5; 789 mod 5672 ; 77 mod 11 ; 55 mod 7 ; 72 mod 13}

\item{5. Soient a, b, n, m ∈ N tels que n ≥ 2, m ≥ 2 et n$|$m. \\
 Prouvez que si a $≡_{m}$ b alors a $≡_{n}$ b.}

\item {6. Soit n ∈ N. Prouvez que si n est impair alors $n^{2} ≡_{2} 1$.}

\item {7. Soient a, b, c, d ∈ Z, m ∈ N, m $>$ 0. \\
    Prouvez que si a $≡_{m}$ b et c $≡_{m}$ d alors (a + c) $≡_{m}$ (b + d) et (ac) $≡_{m}$ (bd).}

\item {8. Déduire de l’exercice précédent que :\\
      (a + b) mod m = ((a mod m) + (b mod m)) mod m \\
      (ab) mod m = ((a mod m)(b mod m)) mod m }

\item {9. Prouvez que les deux égalités suivantes sont fausses : \\
      (a + b) mod m = (a mod m) + (b mod m) \\
      (ab) mod m = (a mod m)(b mod m)}
\end{itemize}


\vspace{3mm} %5mm vertical space
\section{Exercices PPCM-PGCD}
\vspace{3mm} %5mm vertical space

\begin{itemize}

\item {1. Soient m, n ∈ Z et p un nombre premier. Prouvez que si p$|$mn alors p$|$m ou p$|$n. Ce résultat est-il toujours vrai si p n’est pas premier ? }

\item {2. Déterminez lesquels de ces nombres sont premiers : 21, 71, 111 et 143.}

\item {3. Décomposez les nombres suivants en facteurs premiers : 88, 124, 289 et 402.}

\item {4. Calculez pgcd(15, 36), ppcm(21, 49), pgcd(121, 125) et ppcm(31, 81).}

\item{5. Prouvez que le produit de trois entiers consécutifs est toujours divisible par 6.}

\item {6. Écrire en notation binaire les nombres suivants : 7, 9, 11, 31 et 65.}

\item {7. Écrire en notation hexadécimale les nombres suivants : 13, 31 et 65.}

\end{itemize}


\vspace{3mm} %5mm vertical space
\section{Exercices changement de base}
\vspace{3mm} %5mm vertical space

\begin{itemize}

\item {1. Convertir les entiers suivants de l’hexadécimal au décimal : A0B1 et F 0A02. }

\item {2. Convertir les entiers suivants de l’hexadécimal au binaire : ABBA et FACE .}

\item {3. Convertir les entiers suivants du binaire en hexadécimal : 11111011 et 10011101.}

\item {4. Prouvez qu’un nombre entier est divisible par 3 si et seulement si la somme de ses digits en décimal est divisible par 3.}

\item {5. Trouvez l’inverse de 5 modulo 11 ainsi que l’inverse de 3 modulo 7.}

\item {6. Prouvez que 2 240 mod 11 = 1}

\end{itemize}

\newpage
\textbf{1. Prouver que si a ∈ N, a $>$ 0, alors 1 divise a et a divise 0} \\

Cela peut s'écrire simplement : soit a un naturel $>$ 0 $=>$ $\frac{1}{a}$ et $\frac{a}{0}$ \\

\textbf{première partie} \\

$\frac{1}{a}$ $<=>$  a = 1*x (par définition) \\

$\frac{a}{0}$ $<=>$  0 = a*x (par définition) \\

\textbf{deuxième partie} \\

il faut donc lorsque a est posé, trouver un x et un y qui vérifie les expressions suivantes : \\

Soit a $>$ 0, on sait que a un naturel, prenons x=a \\
on a bien a = 1 x a, ce qui vérifie $\frac{1}{a}$ \\

Soit a $>$ 0, on sait que a un naturel, prenons y=0 \\
on a bien 0 = a*0, ce qui vérifie $\frac{a}{0}$

\vspace{10mm}
\textbf{2. Soient a, b ∈ Z, a$\ne$0, b$\ne$0, prouvez que si a$|$b et b$|$a alors a=b ou a=−b} \\

soit a, b un entier positif ou négatif, si a$|$b et b$|$a $=>$ a=b ou a=−b \\

\textbf{première partie} \\

$\frac{a}{b}$ $<=>$  b = a*x (par définition) \\

$\frac{b}{a}$ $<=>$  a = b*y (par définition) \\

\textbf{deuxième partie} \\

il faut donc lorsque a est posé, trouver un x et un y qui vérifie les expressions suivantes : \\

Soit b $\ne$ 0, on sait que b un entier positif ou négatif, prenons x=1 \\

b = a*1 \\

b=a $<=>$ a=b \\
ce qui vérifie l'expressions a=b \\

Soit a $\ne$ 0, on sait que a un entier positif ou négatif, prenons y=-1 \\

a = b*-1 \\
a=-b \\
ce qui vérifie l'expressions a=-b \\

\newpage
\vspace{10mm}
\textbf{3. Déterminez le quotient et le reste de 111 divisé par 11; 123 par 7; 777 divisé par 21; 1434 divisé par 13 et 1025 divisé par 15.}

\vspace{4mm}

$\frac{111}{11}$  Q=10  R=1 \\

$\frac{123}{16}$  Q=17  R=4 \\

$\frac{777}{21}$  Q=37  R=0 \\

$\frac{1437}{13}$ Q=110 R=7 \\

$\frac{1025}{15}$ Q=68  R=5 \\

\vspace{10mm}
\textbf{4. Calculez 7 mod 5; 789 mod 5672; 77 mod 11; 55 mod 7; 72 mod 13}

\vspace{4mm}

7 mod 5 = 2 \\

789 mod 5672 = 789 \\

77 mod 11 = 0 \\

55 mod 7 = 6 \\

72 mod 13 = 7 \\

\vspace{10mm}
\textbf{5. Soient a, b, n, m ∈ N tels que n ≥ 2, m ≥ 2 et n $|$ m. Prouvez que si a $≡_{m}$ b alors a $≡_{n}$ b}

\vspace{4mm}
a = b + k*n (par définition) \\

$\frac{a=k*n+b}{c= l*n + b}$ \\


\vspace{10mm}
\textbf{6. Soit n ∈ N. Prouvez que si n est impair alors $n^{2} ≡_{2} 1$}

\vspace{4mm}
impair = 2x + 1 (par définition) \\

$n^{2}$ = (2x+1)² = $4x^{2}$ + 2*2x*1 + $1^{2}$ \\
$4x^{2}$ + 4x + 1 \\

$4x^{2}$ est pair \\
4x est pair \\
1 est impair \\

PAIR + PAIR + IMPAIR = IMPAIR \\

\vspace{6mm}
\textbf{7. Soient a, b, c, d ∈ Z, m ∈ N, m $>$ 0. Prouvez que si} \\

a $≡_{m}$ b et c $≡_{m}$ d alors (a + c) $≡_{m}$ (b + d) et (ac) $≡_{m}$ (bd) \\

a = b + k*n (par définition) \\

\textbf{première expression} \\

(a + c) $≡_{m}$ (b + d) \\
(a + c) = (m ∗ k + b) + (m ∗ l + d) \\
m * lk + (b + d) \\
(b + d) + m * lk  \\

on peut dire que y = lk et x=(b+d), ce qui vérifie l'expression : a = x + m*y  \\


\textbf{deuxième expression} \\

(ac) $≡_{m}$ (bd) \\
(ac) = (mk + b) ∗ (ml + d) \\
(bd) + m ∗ (mkl + ld + bl)  \\

on peut dire que y = (mkl + ld + bl) et x= b*d, ce qui vérifie l'expression : x + m∗y \\
\newpage

\textbf{8. Déduire de l’exercice précédent que} \\

    (a + b) mod m = ((a mod m) + (b mod m)) mod m \\
    (ab) mod m = ((a mod m)(b mod m)) mod m \\

a $≡_{m}$ b $<=>$ a= b + m*k \\
(a + b) mod m = ((a mod m) + (b mod m)) mod m \\

\textbf{première partie} \\

(a + b) mod m \\
(2b) mod m \\

\textbf{deuxième partie} \\

((a mod m) + (b mod m)) mod m  \\
((b + mk) mod m) (b mod m)) mod m \\

(b + mk) mod m) = b \\
(b + (b mod m)) mod m \\

(b mod m) = b \\
(b + b) mod m \\
(2b) mod m \\

\textbf{Conclusion La première partie == la deuxième partie $=>$ OK} \\

(ab) mod m = ((a mod m)(b mod m)) mod m \\

\textbf{première partie} \\

(ab) mod m \\
((mk + b) b) mod m \\
(mkb + $b^{2}$) mod m \\
$b^{2}$ mod m \\

\textbf{deuxième partie} \\

((a mod m)(b mod m)) mod m \\

(a mod m) = b \\
(b mod m) = b \\

(b*b) mod m \\
($b^{2}$) mod m \\

\textbf{Conclusion La première partie == la deuxième partie $=>$ OK} \\


\newpage
\textbf{9. Prouvez que les deux égalités suivantes sont fausses}\\

      (a + b) mod m = (a mod m) + (b mod m) \\
      (ab) mod m = (a mod m)(b mod m) \\

(a + b) mod m = (a mod m) + (b mod m) \\

\textbf{première partie} \\

(a+b) mod m \\
(b+b) mod m \\
(2b) mod m \\

\textbf{deuxième partie} \\

((a mod m) + (b mod m)) mod m  \\
((b + mk) mod m) (b mod m)) mod m \\

(b + mk) mod m) = b \\
(b + (b mod m)) mod m \\

(b mod m) = b \\
(b + b) = 2b \\

\textbf{Conclusion : la première parties est différentes de la deuxième (2b $\ne$ 2b mod m) } \\

\vspace{4mm}

(ab) mod m = (a mod m)(b mod m) \\

\textbf{première partie} \\

(ab) mod m \\
((mk + b) b) mod m \\
(mkb + $b^{2}$) mod m \\
$b^{2}$ mod m \\

\textbf{deuxième partie} \\

((a mod m)(b mod m)) mod m \\

(a mod m) = b \\
(b mod m) = b \\

(b*b) = $b^{2}$ \\

\textbf{Conclusion : la première parties est différentes de la deuxième ($b^{2}$ mod m $\ne$ $b^{2}$) } \\

\newpage
\textbf{1. Soient m, n ∈ Z et p un nombre premier. Prouvez que si p$|$mn alors p$|$m ou p$|$n. Ce résultat est-il toujours vrai si p n’est pas premier ? }

Nombre premier = nombre divisible par 1 et par lui-même \\

Donc P=x*y car x=$\frac{P}{y}$ y=$\frac{P}{x}$ \\

Vérifier si $\frac{p}{m*n}$ $=>$  p$|$m ou p$|$n \\

p = m*n \\
m=$\frac{P}{n}$ \\
n=$\frac{P}{m}$ \\

ce qui vérifie l'expression P=x*y et donc n'est pas premier \\

\textbf{2. Déterminez lesquels de ces nombres sont premiers : 21, 71, 111 et 143} \\

a) 21 est divisible par 3,7,21 $=>$ NON \\
b) 71 est divisible par 71,1 $=>$ OUI  \\
c) 111 est divisible par 3,111,1 $=>$ NON  \\
d) 143 est divisible par 143,11,1 $=>$ NON \\

\textbf{3. Décomposez les nombres suivants en facteurs premiers : 88, 124, 289 et 402} \\

a) 88 = 2*2*2*11 = 2$^{3}$ * 11 \\
b) 124 = 2*2*31 = 2$^{2}$ * 31 \\
c) 289 = 17$^{2}$ \\
d) 402 = 2*3*67 \\

\textbf{4. Calculez pgcd(15, 36), ppcm(21, 49), pgcd(121, 125) et ppcm(31, 81)} \\

A) pgcd(15, 36)\\
15=3*5 31=31 \\
pgcd(15, 36) = 3 \\

b) ppcm(21, 49) \\
21=3*7  49=7$^{2}$ \\
ppcm(21, 49) = 147 \\

c) pgcd(121, 125) \\
121 = 1*121  125 = 1*5*5*5 \\
pgcd(121, 125) = 1 \\

d) ppcm(31, 81) \\
31=1*31 81=1*81 \\
ppcm(31, 81) = 31*81 = 2511 \\


\textbf{5. Prouvez que le produit de trois entiers consécutifs est toujours divisible par 6.} \\

$\frac{n(n+1)(n+2)} {6}$ \\

$\frac{6n^{3} + 18n^{2} + 12n}{6}$ \\

$n^{3}$ + $3n^{2}$ + 2n \\

il faut qu'il est la forme $6*n$ ou $2*3*n$ \\

\textbf{Divisible par 2} \\

Si n est pair alors $n^{3}$ est pair sinon il est impair. \\

2n$^{2}$ est pair, ce qui donne un nombre pair \\


\textbf{Divisible par 3} \\

Montrer que $n^{3}$ + $3n^{2}$ + 2n est divisible par 3 : \\

Soit $\frac{n}{3}$ Sinon il manque 1 ou 2 unités. \\

Donc n+1 ou n+2 sera divisible par 3 d'où leur produit est divisible par 3.\\


\textbf{6. Écrire en notation binaire les nombres suivants : 7, 9, 11, 31 et 65} \\

7 = 2$^{2}$ + 2$^{1}$ + 2$^{0}$ = 0111 \\

9 = 2$^{3}$ + 2$^{0}$ = 1001 \\

11 = 2$^{6}$ + 2$^{0}$ = 1001 \\

31 = 2$^{4}$ + 2$^{3}$ + 2$^{2}$ + 2$^{1}$ + 2$^{0}$ = 0001 1111 \\

65 = 2$^{6}$ + 2$^{0}$ = 0010 0001 \\

\textbf{7. Écrire en notation hexadécimale les nombres suivants : 13, 31 et 65} \\

13 = 2$^{3}$ + 2$^{2}$ + 2$^{0}$ = 1101 = D \\

31 = 2$^{4}$ + 2$^{3}$ + 2$^{2}$ + 2$^{1}$ + 2$^{0}$ = 0001 1111 = 1F \\

65 = 2$^{6}$ + 2$^{0}$ = 0010 0001 = 41 \\

\newpage

\textbf{1. Convertir les entiers suivants de l’hexadécimal au décimal : A0B1 et F0A02} \\

A0B1 = 10 0 11 1 \\

F0A02 = 15 0 10 0 2 \\

\textbf{2. Convertir les entiers suivants de l’hexadécimal au binaire : ABBA et FACE} \\

ABBA = 1010 1011 1011 1010 \\

FACE = 1111 1010 1100 1110 \\

\textbf{3. Convertir les entiers suivants du binaire en hexadécimal : 11111011 et 10011101} \\

1111 1011 = FB \\

1001 1101 = 9D \\

\textbf{4. Prouvez qu’un nombre entier est divisible par 3 si et seulement si la somme de ses digits en décimal est divisible par 3} \\

Montrer que la sommes des 3 digits est divisible par 3 \\

Z= $\sum_{i>0} a+b+c=3 $ \\
Z=3x ou x ≥ 1 \\
$\frac{z}{3}$ = x \\

tout nombre ayant la sommes des digits est divisible par 3 car \\
tout nombre multiplié par 3 ou x ≥ 1 est divisible par 3 \\


\textbf{5. Trouvez l’inverse de 5 modulo 11 ainsi que l’inverse de 3 modulo 7} \\

5 modulo 11 = 2 et 11-2 = 9 \\

3 modulo 7 = 2 et 7-2 = 5 \\

\newpage
\textbf{6. Prouvez que 2 240 mod 11 = 1} \\

2$^{240}$ = 2$^{2*2*2*2*15}$ \\

$(((((b^{2})^{2})^{2})^{2}) mod 11)^{15}$ \\

$b^{15} mod 11$ \\

Prenons b=9 \\

$9^{15} mod 11$ \\

(9 * ($9^{15}$)) mod 11 \\

(9 * (($9^{2}$)$^{14}$)) mod 11 \\

(9 * (($9^{2}$ mod 11)$^{7}$) mod 11) mod 11 \\

$9^{2}$ mod 11 = 4 \\

(9 * (4* (4$^{6}$ mod 11) mod11) mod 11 \\

(9 * (4* ((4$^{2}$ mod 11)$^{3}$ mod 11) mod11) mod 11 \\

4$^{2}$ mod 11 = 5 \\

(9 * (4* (5$^{3}$ mod 11) mod 11) mod 11 \\

(9 * (4* (5$^{3}$ mod 11) mod 11) mod 11 \\

5$^{3}$ mod 11 = 4 \\

(9 * (4*4 mod 11) mod 11 \\

(9 * (16 mod 11) mod 11 \\

(9 * 5) mod 11 \\

(45) mod 11 = 1 \\
