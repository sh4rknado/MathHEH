\newpage

\chapter{Nombre Entiers Exercices}

\vspace{3mm} %5mm vertical space
\section{Exercices Modulo}
\vspace{3mm} %5mm vertical space

\begin{itemize}

\item {1. Prouver que si a ∈ N, a $>$ 0, alors 1 divise a et a divise 0. }

\item {2. Soient a, b ∈ Z, a 6 = 0, b 6 = 0, prouvez que si a$|$b et b$|$a alors a = b ou a = −b.}

\item {3. Déterminez le quotient et le reste de 111 divisé par 11; 123 par 7; 777
divisé par 21; 1434 divisé par 13 et 1025 divisé par 15.}

\item {4.Calculez 7 mod 5; 789 mod 5672 ; 77 mod 11 ; 55 mod 7 ; 72 mod 13}

\item{5. Soient a, b, n, m ∈ N tels que n ≥ 2, m ≥ 2etn|m. Prouvez que si a ≡ m b alors a ≡ n b.}

\item {6. Soit n ∈ N. Prouvez que si n est impair alors n 2 ≡ 2 1.}

\item {7. Soient a, b, c, d ∈ Z, m ∈ N, m > 0. Prouvez que si a ≡ m b et c ≡ m d alors (a + c) ≡ m (b + d) et (ac) ≡ m (bd).}

\item {8. Déduire de l’exercice précédent que :\\
      (a + b) mod m = ((a mod m) + (b mod m)) mod m \\
      (ab) mod m = ((a mod m)(b mod m)) mod m }

\item {9. Prouvez que les deux égalités suivantes sont fausses : \\
      (a + b) mod m = (a mod m) + (b mod m) \\
      (ab) mod m = (a mod m)(b mod m)}
\end{itemize}


\vspace{3mm} %5mm vertical space
\section{Exercices PPCM-PGCD}
\vspace{3mm} %5mm vertical space

\begin{itemize}

\item {1. Soient m, n ∈ Z et p un nombre premier. Prouvez que si p$|$mn alors p$|$m ou p$|$n. Ce résultat est-il toujours vrai si p n’est pas premier ? }

\item {2. Déterminez lesquels de ces nombres sont premiers : 21, 71, 111 et 143.}

\item {3. Décomposez les nombres suivants en facteurs premiers : 88, 124, 289 et 402.}

\item {4. Calculez pgcd(15, 36), ppcm(21, 49), pgcd(121, 125) et ppcm(31, 81).}

\item{5. Prouvez que le produit de trois entiers consécutifs est toujours divisible par 6.}

\item {6. Écrire en notation binaire les nombres suivants : 7, 9, 11, 31 et 65.}

\item {7. Écrire en notation hexadécimale les nombres suivants : 13, 31 et 65.}

\end{itemize}


\vspace{3mm} %5mm vertical space
\section{Exercices changement de base}
\vspace{3mm} %5mm vertical space

\begin{itemize}

\item {1. Convertir les entiers suivants de l’hexadécimal au décimal : A0B1 et F 0A02. }

\item {2. Convertir les entiers suivants de l’hexadécimal au binaire : ABBA et FACE .}

\item {3. Convertir les entiers suivants du binaire en hexadécimal : 11111011 et 10011101.}

\item {4. Prouvez qu’un nombre entier est divisible par 3 si et seulement si la somme de ses digits en décimal est divisible par 3.}

\item {5. Trouvez l’inverse de 5 modulo 11 ainsi que l’inverse de 3 modulo 7.}

\item {6. Prouvez que 2 240 mod 11 = 1}

\end{itemize}
