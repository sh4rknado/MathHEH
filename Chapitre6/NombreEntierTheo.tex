\newpage

\chapter{Nombre Entiers Théorie}
\vspace{3mm} %5mm vertical space
\section{Modulo}
\vspace{3mm} %5mm vertical space

Le modulo est le reste de la division entière de A par B. \\

\vspace{3mm} %5mm vertical space
\textbf{Modulo exemple}
\vspace{3mm} %5mm vertical space

Soient a,b et m des nombre naturels. Est-ce que (a+b) mod m = ((a mod m)+(b mod m)) mod m \\

(8+15) mod 3 = ((8 mod 3)+(15 mod 3)) mod 3 \\
(23) mod 3 = (2+0) mod 3 \\
2 = 2

\vspace{3mm} %5mm vertical space
\section{Transformation}
\vspace{3mm} %5mm vertical space

Il faudra tester les un après les autres les nombre premier par ordre croissant. \\

(2,3,5,7,11,13,17)

\vspace{3mm} %5mm vertical space
\textbf{exemple de la décomposition}
\vspace{3mm} %5mm vertical space

tout nombre se finissant par 0 est divisible par 2 \\

17640/2 \\
8820/2  \\
4410/2  \\
2205/3 $=>$ 5+2+2 = 9 et 9 est divisible par 3 \\
735/3 $=>$ 7+3+5 = 15 et 15 est divisible par 3 \\
245/5 $=>$ 2+4+5 = 11 mais 245 est divisible par 5 \\
49/7 \\
7/7 \\
1

17640 = 2*2*2*3*3*3*5*7*7 ou $2^{3}*3^{2}*5*7^{2}$ \\


411600/2 \\
205800/2 \\
102900/2 \\
51450/2 \\
25725/3 $=>$ 2+5+7+2+5 = 21 et 21 est divisible par 3 \\
8575/5 $=>$ 8+5+7+5 = 25 et 25 est divisible par 5 \\
1715/5 $=>$ est divisible par 5 \\
343/7 $=>$ 343=$7^{4}$ est divisible par 7 \\
49/7 \\
7/7 \\
1

411600 = 2*2*2*2*5*5*7*7*7 ou $2^{4}*3*5^{2}*7^{3}$ \\

\section{PGCD}

il faut toujours prendre l'exposant le plus petit pour être sur de pouvoir être divisible par les 2 nombres. \\

17640 = $2^{3}*3^{2}*5*7^{2}$ \\
411600 = $2^{4}*3*5^{2}*7^{3}$ \\

\textbf{PGCD = $2^{3}*3*5*7^{2}$ = 5880} \\

\section{PPCM}

il faut toujours prendre l'exposant le plus grand pour être sur qu'il soit un multiple des 2 nombres. \\

17640 = $2^{3}*3^{2}*5*7^{2}$ \\
411600 = $2^{4}*3*5^{2}*7^{3}$ \\

\textbf{PPCM = $2^{4}*3^{2}*5^{2}*7^{3}$ = 1 234 800 } \\
