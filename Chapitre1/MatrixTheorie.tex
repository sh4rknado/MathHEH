
\newpage
\section{Matrices Théories}
\vspace{10mm} %5mm vertical space

\subsection{Les propriétés}
\vspace{5mm} %5mm vertical space

A) Linéarité \\

si on multiplie une matrice par $\lambda$, le déterminant est multiplié par $\lambda^{n}$ et toutes les lignes et colonnes sont multiplié par $\lambda=det(A)*\lambda^{n}$ \\
\vspace{5mm} %5mm vertical space

$
det(A+B) \neq det(A)+det(B) ? \\
$

Exemple:\\

$
A =
\begin{pmatrix}
  a & 0 \\
  0 & b \\
\end{pmatrix}
$
\vspace{5mm} %5mm vertical space
$
B =
\begin{pmatrix}
  c & 0 \\
  0 & d \\
\end{pmatrix}
$

\vspace{3mm} %5mm vertical space
det(A)=ab et det(B)=cd\\
\vspace{5mm} %5mm vertical space

Conclusion : \\

$
C =
\begin{pmatrix}
  a+c & 0 \\
  0 & b+d \\
\end{pmatrix}
$

\vspace{3mm} %5mm vertical space
$
det(C)=(a+c)*(b+d) \\
$

\vspace{3mm} %5mm vertical space
$\lambda^{n} \neq$ linéaire \\
$\lambda^{n}$ est exponentielle

\vspace{10mm} %5mm vertical space
B) Déterminant et transposée \\

Det(A) = det(A), les déterminants sont égaux, il y a juste la signature (le signe) qui est modifiée.\\

Démonstration :\\

$det(A) = \sum_{o \varepsilon s} \varepsilon (o^{-1}), ...$\\

$det(T_{a}) = \sum_{o \varepsilon s} \varepsilon (o^{1}), ...$

\vspace{10mm} %5mm vertical space
C) Déterminant et produit \\

les déterminants sont compatible avec le produit det(AB) = det(A) * det(B) \\

$
\varphi_{a} (x_{1}, ..., x_{n}) = det(\varphi_{c}) (A*1, ..., A*N))
$

\vspace{10mm} %5mm vertical space
D) Déterminant et matrice inversible \\

Une matrice est inversible uniquement si le déterminant est différents de 0. \\

$
det(A^{-1}) = \frac{1}{det(A)}
$

\subsection{Calcul du déterminants 2*2}
\vspace{5mm} %5mm vertical space
Le calcul du déterminants d'une matrice 2*2 est le résultat d'une soustraction entre la multiplications croisée des 2 ensembles \\
Il faut utiliser la ligne avec le plus de 0.

\vspace{5mm} %5mm vertical space

$
A =
\begin{pmatrix}
  1 & 4 \\
  2 & 3 \\
\end{pmatrix}
$

\vspace{5mm} %5mm vertical space

det(A) = (1*3) - (2*4)\\

det(A) = (3-8)\\

det(A) = (-5)\\

S = -5\\

\subsection{Calcul du déterminants 4*4 ou n*n}
\vspace{5mm} %5mm vertical space
Le calcul du déterminants d'une matrice n*n est le résultat d'une série d'opération entre les sous matrices.
\vspace{5mm} %5mm vertical space

$
A =
\begin{pmatrix}
  0 & 1 & 2 & 3 \\
  1 & 2 & 3 & 0 \\
  2 & 3 & 0 & 1 \\
  3 & 0 & 1 & 2 \\
\end{pmatrix}
$

\vspace{10mm} %5mm vertical space

Extraction des sous matrices \\

\vspace{5mm} %5mm vertical space
Matrices de signes

\vspace{4mm} %5mm vertical space

$
\begin{pmatrix}
  + & - & + & -\\
  - & + & - & +\\
  + & - & + & -\\
  - & + & - & +\\
\end{pmatrix}
$

\vspace{8mm} %5mm vertical space
Extraction des matrices 3*3
\vspace{5mm} %5mm vertical space

$
1*(
  +1*
  \begin{pmatrix}
    1 & 2 & 1 \\
    0 & 0 & 1 \\
    1 & 2 & 3 \\
  \end{pmatrix}
  $
  $
  +3*
  \begin{pmatrix}
    0 & 1 & 1 \\
    3 & 0 & 1 \\
    0 & 1 & 3 \\
  \end{pmatrix}
  $
  $
  -4*
  \begin{pmatrix}
    0 & 1 & 2 \\
    3 & 0 & 0 \\
    0 & 1 & 2 \\
  \end{pmatrix}
  $
)

\vspace{8mm} %5mm vertical space
Extraction des matrices 2*2
\vspace{5mm} %5mm vertical space

$
1*(1*(
+1*
\begin{pmatrix}
  0 & 1 \\
  2 & 3 \\
\end{pmatrix}
$
$
-2*
\begin{pmatrix}
  0 & 1 \\
  1 & 3 \\
\end{pmatrix}
$
$
+1*
\begin{pmatrix}
  0 & 0 \\
  1 & 2 \\
\end{pmatrix}
$
)
  $
  +3*(
  -1*
  \begin{pmatrix}
    3 & 1 \\
    0 & 3 \\
  \end{pmatrix}
  $
  $
  +1*
  \begin{pmatrix}
    3 & 0 \\
    0 & 1 \\
  \end{pmatrix}
  $
  )
  $
  -4*(
  -1*
  \begin{pmatrix}
    3 & 0 \\
    0 & 2 \\
  \end{pmatrix}
  $
  $
  +2*
  \begin{pmatrix}
    3 & 0 \\
    0 & 1 \\
  \end{pmatrix}
  $
  )
)

\newpage
\vspace{8mm} %5mm vertical space
Calcul du déterminants des matrices 2*2
\vspace{5mm} %5mm vertical space

1*(\\
  1*($+1*( (0*3) - (1*2)) -2*( (0*3) - (1*1)) +1*( (0*2) - (0*1)))$ \\
  +3*($ -1((3*3) - (0*1)) +1((3*1) - (0*0))))$ \\
  -4*($ -1((3*2) - (0*0)) +2((3*1) - (0*0)))$ \\
)\\



\vspace{4mm} %5mm vertical space
Simplification de l'équation
\vspace{5mm} %5mm vertical space

$
1*(\\
  1*(+1*(0 - 2)-2*(0 - 1)+1*(0 - 0))\\
  +3*(-1(9 - 0)+1(3 - 0))\\
  -4*(-1(6 - 0)+2(3 - 0))\\
  )
$


\vspace{4mm} %5mm vertical space
Mise en équations et résolution
\vspace{5mm} %5mm vertical space

$ 1*(1*((-2)-(-2)) +3*(-9 + 3) -4*(-6 +6 ))$

1*(3*(-9 + 3)) = -27+9 = -18\\

det(A) = -18\\
