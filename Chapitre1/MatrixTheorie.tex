
\newpage
\section{Matrices Théories}
\vspace{10mm} %5mm vertical space

\subsection{Les propriétés}
\vspace{5mm} %5mm vertical space

A) Linéarité \\

si on multiplie une matrice par $\lambda$, le déterminant est multiplié par $\lambda^{n}$ et toutes les lignes et colonnes sont multiplié par $\lambda=det(A)*\lambda^{n}$ \\
\vspace{5mm} %5mm vertical space

$
det(A+B) \neq det(A)+det(B) ? \\
$

Exemple:\\

$
A =
\begin{pmatrix}
  a & 0 \\
  0 & b \\
\end{pmatrix}
$
\vspace{5mm} %5mm vertical space
$
B =
\begin{pmatrix}
  c & 0 \\
  0 & d \\
\end{pmatrix}
$

\vspace{3mm} %5mm vertical space
det(A)=ab et det(B)=cd\\
\vspace{5mm} %5mm vertical space

Conclusion : \\

$
C =
\begin{pmatrix}
  a+c & 0 \\
  0 & b+d \\
\end{pmatrix}
$

\vspace{3mm} %5mm vertical space
$
det(C)=(a+c)*(b+d) \\
$

\vspace{3mm} %5mm vertical space
$\lambda^{n} \neq$ linéaire \\
$\lambda^{n}$ est exponentielle

\vspace{10mm} %5mm vertical space
B) Déterminant et transposée \\

Det(A) = det(A), les déterminants sont égaux, il y a juste la signature (le signe) qui est modifiée.\\

Démonstration :\\

$det(A) = \sum_{o \varepsilon s} \varepsilon (o^{-1}), ...$\\

$det(T_{a}) = \sum_{o \varepsilon s} \varepsilon (o^{1}), ...$

\vspace{10mm} %5mm vertical space
C) Déterminant et produit \\

les déterminants sont compatible avec le produit det(AB) = det(A) * det(B) \\

$
\varphi_{a} (x_{1}, ..., x_{n}) = det(\varphi_{c}) (A*1, ..., A*N))
$

\vspace{10mm} %5mm vertical space
D) Déterminant et matrice inversible \\

Une matrice est inversible uniquement si le déterminant est différents de 0. \\

$
det(A^{-1}) = \frac{1}{det(A)}
$

\subsection{Calcul du déterminants 2*2}
\vspace{5mm} %5mm vertical space
Le calcul du déterminants d'une matrice 2*2 est le résultat d'une soustraction entre la multiplications croisée des 2 ensembles \\
Il faut utiliser la ligne avec le plus de 0.

\vspace{5mm} %5mm vertical space

$
A =
\begin{pmatrix}
  1 & 4 \\
  2 & 3 \\
\end{pmatrix}
$

\vspace{5mm} %5mm vertical space

det(A) = (1*3) - (2*4)\\

det(A) = (3-8)\\

det(A) = (-5)\\

S = -5\\

\subsection{Calcul du déterminants 4*4 ou n*n}
\vspace{5mm} %5mm vertical space
Le calcul du déterminants d'une matrice n*n est le résultat d'une série d'opération entre les sous matrices.
\vspace{5mm} %5mm vertical space

$
A =
\begin{pmatrix}
  0 & 1 & 2 & 3 \\
  1 & 2 & 3 & 0 \\
  2 & 3 & 0 & 1 \\
  3 & 0 & 1 & 2 \\
\end{pmatrix}
$

\vspace{10mm} %5mm vertical space


$
A =
\begin{pmatrix}
  1 & 2 & 3 & 0 \\
  0 & 1 & 2 & 3 \\
  2 & 3 & 0 & 1 \\
  3 & 0 & 1 & 2 \\
\end{pmatrix}
$

\vspace{10mm} %5mm vertical space
Inversion de L1 avec L2\\

$
A =
\begin{pmatrix}
  \hl{\textbf{0}} & \hl{\textbf{1}} & \hl{\textbf{2}} & \hl{\textbf{3}} \\
  \hl{\textbf{1}} & \hl{\textbf{2}} & \hl{\textbf{3}} & \hl{\textbf{0}} \\
  2 & 3 & 0 & 1 \\
  3 & 0 & 1 & 2 \\
\end{pmatrix}
$
\vspace{5mm} %5mm vertical space

$
A =
\begin{pmatrix}
  \hl{\textbf{1}} & \hl{\textbf{2}} & \hl{\textbf{3}} & \hl{\textbf{0}} \\
  \hl{\textbf{0}} & \hl{\textbf{-1}} & \hl{\textbf{-2}} & \hl{\textbf{-3}} \\
  2 & 3 & 0 & 1 \\
  3 & 0 & 1 & 2 \\
\end{pmatrix}
$

\vspace{10mm} %5mm vertical space
Méthodes du pivot de Gauss \\

\vspace{10mm} %5mm vertical space
Mise à zero de L3 \\
L3 - (2*L1) = L3 \\

$
A =
\begin{pmatrix}
  1 & 2 & 3 & 0 \\
  0 & -1 & -2 & -3 \\
  \hl{\textbf{2-(1*2)}} & \hl{\textbf{3-(2*2)}} & \hl{\textbf{0-(2*3)}} & \hl{\textbf{1-(2*0)}} \\
  3 & 0 & 1 & 2 \\
\end{pmatrix}
$

\vspace{5mm} %5mm vertical space

$
A =
\begin{pmatrix}
  1 & 2 & 3 & 0 \\
  0 & -1 & -2 & -3 \\
  \hl{\textbf{(2-2)}} & \hl{\textbf{3-4}} & \hl{\textbf{(0-6)}} & \hl{\textbf{1-0)}} \\
  3 & 0 & 1 & 2 \\
\end{pmatrix}
$

\vspace{5mm} %5mm vertical space

$
A =
\begin{pmatrix}
  \hl{\textbf{1}} & 2 & 3 & 0 \\
  0 & -1 & -2 & -3 \\
  0 & -1 & -6 & 1 \\
  3 & 0 & 1 & 2 \\
\end{pmatrix}
$

\vspace{10mm} %5mm vertical space

Mise à zero de L4 \\
L4 - (3*L1) = L4 \\

$
A =
\begin{pmatrix}
  \hl{\textbf{1}} & 2 & 3 & 0 \\
  0 & -1 & -2 & -3 \\
  0 & -1 & -6 & 1 \\
  \hl{\textbf{3-(3*1)}} & \hl{\textbf{0-(3*2)}} & \hl{\textbf{1-(3*3)}} & \hl{\textbf{2-(3*0)}} \\
\end{pmatrix}
$

\vspace{5mm} %5mm vertical space

$
A =
\begin{pmatrix}
  \hl{\textbf{1}} & 2 & 3 & 0 \\
  0 & -1 & -2 & -3 \\
  0 & -1 & -6 & 1 \\
  \hl{\textbf{3-3}} & \hl{\textbf{0-6}} & \hl{\textbf{1-9}} & \hl{\textbf{2-0}} \\
\end{pmatrix}
$

\vspace{5mm} %5mm vertical space

$
A =
\begin{pmatrix}
  1 & 2 & 3 & 0   \\
  0 & \hl{\textbf{-1}} & -2 & -3   \\
  0 & -1 & -6 & 1 \\
  0 & -6 & -8 & 2 \\
\end{pmatrix}
$

\vspace{5mm} %5mm vertical space

L3 = L3-1*L2 \\

$
A =
\begin{pmatrix}
  1 & 2 & 3 & 0    \\
  0 & \hl{\textbf{-1}} & -2 & -3 \\
  0 & \hl{\textbf{0}} & \hl{\textbf{-4}} & \hl{\textbf{4}}   \\
  0 & -6 & -8 & 2  \\
\end{pmatrix}
$

\vspace{5mm} %5mm vertical space

L4 = L4-6*L2 \\

\vspace{5mm} %5mm vertical space

$
A =
\begin{pmatrix}
  1 & 2 & 3 & 0    \\
  0 & \hl{\textbf{-1}} & -2 & -3 \\
  0 & 0 & -4 & 4   \\
  0 & \hl{\textbf{0}} & \hl{\textbf{4}} & \hl{\textbf{20}}   \\
\end{pmatrix}
$

\vspace{5mm} %5mm vertical space

$
A =
\begin{pmatrix}
  1 & 2 & 3 & 0    \\
  0 & -1 & -2 & -3 \\
  0 & 0 & \hl{\textbf{-4}} & 4   \\
  0 & 0 & 4 & 20  \\
\end{pmatrix}
$

\vspace{5mm} %5mm vertical space

L4-(-1)*L3\\

\vspace{5mm} %5mm vertical space

$
A =
\begin{pmatrix}
  1 & 2 & 3 & 0    \\
  0 & -1 & -2 & -3 \\
  0 & 0 & \hl{\textbf{-4}} & 4   \\
  \hl{\textbf{0}} & \hl{\textbf{0}} & \hl{\textbf{0}} & \hl{\textbf{20}}  \\
\end{pmatrix}
$

\vspace{5mm} %5mm vertical space
Fin de la triangulaire Suppérieures \\

$
A =
\begin{pmatrix}
  1 & 2 & 3 & 0    \\
  \hl{\textbf{0}} & -1 & -2 & -3 \\
  \hl{\textbf{0}} & \hl{\textbf{0}} & -4 & 4   \\
  \hl{\textbf{0}} & \hl{\textbf{0}} & \hl{\textbf{0}} & 20  \\
\end{pmatrix}
$

\vspace{5mm} %5mm vertical space

1*(-1)*(-4)*24=\hl{\textbf{96}} \\
\vspace{4mm} %5mm vertical space
S= det(A) = \hl{\textbf{96}} \\
