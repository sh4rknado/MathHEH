
\newpage
\section{Exercice Matrices}
\vspace{10mm} %5mm vertical space
\subsection{Enoncés des exercices}
\vspace{5mm} %5mm vertical space

$
A =
\begin{pmatrix}
  0 & 1 & 2 & 3 \\
  1 & 2 & 3 & 0 \\
  2 & 3 & 0 & 1 \\
  3 & 0 & 1 & 2 \\
\end{pmatrix}
$
\vspace{5mm} %5mm vertical space
$
B =
\begin{pmatrix}
  1 & 4 \\
  2 & 3 \\
  3 & 2 \\
  4 & 1 \\
\end{pmatrix}
$
\vspace{5mm} %5mm vertical space
$
C =
\begin{pmatrix}
  1 & 2 & 3 & 4 \\
  4 & 3 & 2 & 1 \\
\end{pmatrix}
$
\vspace{3mm} %5mm vertical space

\begin{enumerate}[label=\Alph*)]
\item Calculer B*C
\item Calculer la trace de A
\item Calculer la transposée de B
\item Calculer 2,5*C
\item Calculer $B^{t}+C$
\item Exercices supplémentaire (Déplacement 3D)
\item Calculer le déterminants de A
\item Exercices prépartion examen (déterminant)
\item Exercices prépartion examen (déterminant)
\item Exercices prépartion examen (déterminant)
\item Exercices prépartion examen (déterminant)
\end{enumerate}

\newpage

\subsection{Résolution des exercices}
\vspace{5mm} %5mm vertical space
A) Calculer B*C
\vspace{10mm} %5mm vertical space

$
B*C =
\begin{pmatrix}
  1*1+4*4 & 1*2+4*3 & 1*3+4*2 & 1*4+4*1 \\
  2*1+3*4 & 2*2+3*3 & 2*3+3*2 & 2*4+3*1 \\
  3*1+2*4 & 3*2+2*3 & 3*3+2*2 & 3*4+2*1 \\
  4*1+1*4 & 4*2+1*3 & 4*3+1*2 & 4*4+1*1 \\
\end{pmatrix}
$

\vspace{5mm} %5mm vertical space

$
S = B*C =
\begin{pmatrix}
  17 & 14 & 11 & 8 \\
  14 & 13 & 12 & 10 \\
  11 & 12 & 13 & 15 \\
  8 & 11 & 14 & 17 \\
\end{pmatrix}
$

\vspace{10mm} %5mm vertical space
B) Calculer la trace de A
\vspace{5mm} %5mm vertical space

La trace d'une matrices est la somme de chaque éléments de sa diagonale
\vspace{5mm} %5mm vertical space

$
A =
\begin{pmatrix}
  \hl{\textbf{0}}  & 1 & 2 & 3 \\
  1 & \hl{\textbf{2}} & 3 & 0 \\
  2 & 3 & \hl{\textbf{0}} & 1 \\
  3 & 0 & 1 & \hl{\textbf{2}} \\
\end{pmatrix}
$
\vspace{3mm} %5mm vertical space

La trace de la matrice A = 0+2+0+2 = 4 \\
S = 4 \\

\vspace{10mm} %5mm vertical space
C) Calculer la transposée de la matrice B

\vspace{4mm} %5mm vertical space
La transposée de la matrice est d'intervertir les lignes/colonnes de la matrice originale.

\vspace{5mm} %5mm vertical space
$
B =
\begin{pmatrix}
  1 & 4 \\
  2 & 3 \\
  3 & 2 \\
  4 & 1 \\
\end{pmatrix}
$
\vspace{5mm} %5mm vertical space
$
B^{t} =
\begin{pmatrix}
  1 & 2 & 3 & 4 \\
  4 & 3 & 2 & 1 \\
\end{pmatrix}
$

Notes : $B^{t}$ est égale à C \\

$
B^{t} = C =
\begin{pmatrix}
  1 & 2 & 3 & 4 \\
  4 & 3 & 2 & 1 \\
\end{pmatrix}
$

\vspace{5mm} %5mm vertical space

S = $B^{t}$ ou C\\

\vspace{10mm} %5mm vertical space
D) Calculer 2,5*C

\vspace{5mm} %5mm vertical space
$
2,5*C =
\begin{pmatrix}
  1*2,5 & 2*2,5 & 3*2,5 & 4*2,5 \\
  4*2,5 & 3*2,5 & 2*2,5 & 1*2,5 \\
\end{pmatrix}
$
\vspace{5mm} %5mm vertical space

$
S = 2,5*C =
\begin{pmatrix}
  2,5 & 5 & 7,5 & 10 \\
  10 & 7,5 & 5 & 2,5 \\
\end{pmatrix}
$


\vspace{10mm} %5mm vertical space
E) Calculer $B^{t} + C$

\vspace{5mm} %5mm vertical space
$
B^{t} = C =
\begin{pmatrix}
  1 & 2 & 3 & 4 \\
  4 & 3 & 2 & 1 \\
\end{pmatrix}
$
\vspace{5mm} %5mm vertical space

Notes : $B^{t} = C$ = C+C ou 2*C
\vspace{5mm} %5mm vertical space

$
S = 2*C =
\begin{pmatrix}
  1*2 & 2*2 & 3*2 & 4*2 \\
  4*2 & 3*2 & 2*2 & 1*2 \\
\end{pmatrix}
$

\vspace{5mm} %5mm vertical space
S = $ B^{t}$ +C = 2*C = C+C = \\

$
\begin{pmatrix}
  2 & 4 & 6 & 8 \\
  8 & 6 & 4 & 2 \\
\end{pmatrix}
$

\vspace{10mm} %5mm vertical space

F) Déplacement 3D \\

R=10u H=300l où L=40cm + hauteur du casier \\
P=($(\frac{3}{5})* R<R$) \\
$\theta =0$ Z= $R+(\frac{B}{100}*R) = R+(\frac{2}{100})*R=20cm $ \\

Etape 0: Coordonnées de la pince: \\

$
\begin{pmatrix}
  X_{0} \\
  Y_{0} \\
  Z_{0} \\
\end{pmatrix}
$
\vspace{5mm} %5mm vertical space
$
=
\begin{pmatrix}
  \frac{3}{5}R \\
  0 \\
  5l \\
\end{pmatrix}
$

Etape 1: Allongement de la pince: \\

$
\begin{pmatrix}
  X_{1} \\
  Y_{1} \\
  Z_{1} \\
\end{pmatrix}
$
\vspace{5mm} %5mm vertical space
$
=
\begin{pmatrix}
  X_{0} \\
  Y_{0} \\
  Z_{0} \\
\end{pmatrix}
$
\vspace{5mm} %5mm vertical space
$
 +
\begin{pmatrix}
 (\frac{3}{5}R + \frac{13}{110})*R  \\
  0 \\
  0 \\
\end{pmatrix}
$

Etape 2: Rétraction de la pince + marge: \\

$
\begin{pmatrix}
  X_{2} \\
  Y_{2} \\
  Z_{2} \\
\end{pmatrix}
$
\vspace{5mm} %5mm vertical space
$
=
\begin{pmatrix}
  X_{1} \\
  Y_{1} \\
  Z_{1} \\
\end{pmatrix}
$
\vspace{5mm} %5mm vertical space
$
 -
\begin{pmatrix}
 (\frac{R}{2} + \frac{B}{100})*R  \\
  0 \\
  0 \\
\end{pmatrix}
$

Etape 3: Bras monté à 15l : \\

$
\begin{pmatrix}
  X_{3} \\
  Y_{3} \\
  Z_{3} \\
\end{pmatrix}
$
\vspace{5mm} %5mm vertical space
$
=
\begin{pmatrix}
  X_{2} \\
  Y_{2} \\
  Z_{2} \\
\end{pmatrix}
$
\vspace{5mm} %5mm vertical space
$
 +
\begin{pmatrix}
  0 \\
  0 \\
  15l \\
\end{pmatrix}
$

Etape 4: Mouvement à 45° \\

$
\begin{pmatrix}
  X_{4} \\
  Y_{4} \\
  Z_{4} \\
\end{pmatrix}
$
\vspace{5mm} %5mm vertical space
$
=
\begin{pmatrix}
  X_{3} \\
  Y_{3} \\
  Z_{3} \\
\end{pmatrix}
$
\vspace{5mm} %5mm vertical space
$
 +
\begin{pmatrix}
 (cos(45)-sin(45) & 0 \\
  sin(45)-cos(45) & 0\\
  0 & 1 \\
\end{pmatrix}
$

Etape 5: Allongement \\

$
\begin{pmatrix}
  X_{5} \\
  Y_{5} \\
  Z_{5} \\
\end{pmatrix}
$
\vspace{5mm} %5mm vertical space
$
=
\begin{pmatrix}
  X_{4} \\
  Y_{4} \\
  Z_{4} \\
\end{pmatrix}
$
\vspace{5mm} %5mm vertical space
$
 +
\begin{pmatrix}
  (\frac{3}{5}R + \frac{13}{110})*R  \\
  0 \\
  0 \\
\end{pmatrix}
$

Etape 6: Rétraction + marge : \\

$
\begin{pmatrix}
  X_{6} \\
  Y_{6} \\
  Z_{6} \\
\end{pmatrix}
$
\vspace{5mm} %5mm vertical space
$
=
\begin{pmatrix}
  X_{5} \\
  Y_{5} \\
  Z_{5} \\
\end{pmatrix}
$
\vspace{5mm} %5mm vertical space
$
 +
\begin{pmatrix}
  (\frac{R}{2} + \frac{B}{100})*R  \\
  0 \\
  0 \\
\end{pmatrix}
$

Etape 7: Rotation -45° et Retour à 0 : \\

$
\begin{pmatrix}
  X_{7} \\
  Y_{7} \\
  Z_{7} \\
\end{pmatrix}
$
\vspace{5mm} %5mm vertical space
$
=
\begin{pmatrix}
  X_{6} \\
  Y_{6} \\
  Z_{6} \\
\end{pmatrix}
$
\vspace{5mm} %5mm vertical space
$
 -
\begin{pmatrix}
  Cos(45)-sin(45) & 0  \\
  +Sin(45)Cos(45) & 0 \\
  0 & 1 \\
\end{pmatrix}
$
$
 -
\begin{pmatrix}
  0  \\
  0 \\
  -15l \\
\end{pmatrix}
$

\newpage

G) Calcul du déterminant \\

$
A =
\begin{pmatrix}
  1 & 2 & 3 & 0 \\
  0 & 1 & 2 & 3 \\
  2 & 3 & 0 & 1 \\
  3 & 0 & 1 & 2 \\
\end{pmatrix}
$

\vspace{10mm} %5mm vertical space
Inversion de L1 avec L2\\

$
A =
\begin{pmatrix}
  \hl{\textbf{0}} & \hl{\textbf{1}} & \hl{\textbf{2}} & \hl{\textbf{3}} \\
  \hl{\textbf{1}} & \hl{\textbf{2}} & \hl{\textbf{3}} & \hl{\textbf{0}} \\
  2 & 3 & 0 & 1 \\
  3 & 0 & 1 & 2 \\
\end{pmatrix}
$
\vspace{5mm} %5mm vertical space

$
A =
\begin{pmatrix}
  \hl{\textbf{1}} & \hl{\textbf{2}} & \hl{\textbf{3}} & \hl{\textbf{0}} \\
  \hl{\textbf{0}} & \hl{\textbf{-1}} & \hl{\textbf{-2}} & \hl{\textbf{-3}} \\
  2 & 3 & 0 & 1 \\
  3 & 0 & 1 & 2 \\
\end{pmatrix}
$

\vspace{10mm} %5mm vertical space
Méthodes du pivot de Gauss \\

\vspace{5mm} %5mm vertical space
Mise à zero de L3 \\
L3 - (2*L1) = L3 \\

$
A =
\begin{pmatrix}
  1 & 2 & 3 & 0 \\
  0 & -1 & -2 & -3 \\
  \hl{\textbf{2-(1*2)}} & \hl{\textbf{3-(2*2)}} & \hl{\textbf{0-(2*3)}} & \hl{\textbf{1-(2*0)}} \\
  3 & 0 & 1 & 2 \\
\end{pmatrix}
$

\vspace{5mm} %5mm vertical space

$
A =
\begin{pmatrix}
  1 & 2 & 3 & 0 \\
  0 & -1 & -2 & -3 \\
  \hl{\textbf{(2-2)}} & \hl{\textbf{3-4}} & \hl{\textbf{(0-6)}} & \hl{\textbf{1-0)}} \\
  3 & 0 & 1 & 2 \\
\end{pmatrix}
$

\vspace{5mm} %5mm vertical space

$
A =
\begin{pmatrix}
  \hl{\textbf{1}} & 2 & 3 & 0 \\
  0 & -1 & -2 & -3 \\
  0 & -1 & -6 & 1 \\
  3 & 0 & 1 & 2 \\
\end{pmatrix}
$

\vspace{10mm} %5mm vertical space

Mise à zero de L4 \\
L4 - (3*L1) = L4 \\

$
A =
\begin{pmatrix}
  \hl{\textbf{1}} & 2 & 3 & 0 \\
  0 & -1 & -2 & -3 \\
  0 & -1 & -6 & 1 \\
  \hl{\textbf{3-(3*1)}} & \hl{\textbf{0-(3*2)}} & \hl{\textbf{1-(3*3)}} & \hl{\textbf{2-(3*0)}} \\
\end{pmatrix}
$

\vspace{5mm} %5mm vertical space

$
A =
\begin{pmatrix}
  \hl{\textbf{1}} & 2 & 3 & 0 \\
  0 & -1 & -2 & -3 \\
  0 & -1 & -6 & 1 \\
  \hl{\textbf{3-3}} & \hl{\textbf{0-6}} & \hl{\textbf{1-9}} & \hl{\textbf{2-0}} \\
\end{pmatrix}
$

\vspace{5mm} %5mm vertical space

$
A =
\begin{pmatrix}
  1 & 2 & 3 & 0   \\
  0 & \hl{\textbf{-1}} & -2 & -3   \\
  0 & -1 & -6 & 1 \\
  0 & -6 & -8 & 2 \\
\end{pmatrix}
$

\vspace{5mm} %5mm vertical space

L3 = L3-1*L2 \\

$
A =
\begin{pmatrix}
  1 & 2 & 3 & 0    \\
  0 & \hl{\textbf{-1}} & -2 & -3 \\
  0 & \hl{\textbf{0}} & \hl{\textbf{-4}} & \hl{\textbf{4}}   \\
  0 & -6 & -8 & 2  \\
\end{pmatrix}
$

\vspace{5mm} %5mm vertical space

L4 = L4-6*L2 \\

\vspace{5mm} %5mm vertical space

$
A =
\begin{pmatrix}
  1 & 2 & 3 & 0    \\
  0 & \hl{\textbf{-1}} & -2 & -3 \\
  0 & 0 & -4 & 4   \\
  0 & \hl{\textbf{0}} & \hl{\textbf{4}} & \hl{\textbf{20}}   \\
\end{pmatrix}
$

\vspace{5mm} %5mm vertical space

$
A =
\begin{pmatrix}
  1 & 2 & 3 & 0    \\
  0 & -1 & -2 & -3 \\
  0 & 0 & \hl{\textbf{-4}} & 4   \\
  0 & 0 & 4 & 20  \\
\end{pmatrix}
$

\vspace{5mm} %5mm vertical space

L4-(-1)*L3\\

\vspace{5mm} %5mm vertical space

$
A =
\begin{pmatrix}
  1 & 2 & 3 & 0    \\
  0 & -1 & -2 & -3 \\
  0 & 0 & \hl{\textbf{-4}} & 4   \\
  \hl{\textbf{0}} & \hl{\textbf{0}} & \hl{\textbf{0}} & \hl{\textbf{24}}  \\
\end{pmatrix}
$

\vspace{5mm} %5mm vertical space
Fin de la triangulaire Suppérieures \\

$
A =
\begin{pmatrix}
  1 & 2 & 3 & 0    \\
  \hl{\textbf{0}} & -1 & -2 & -3 \\
  \hl{\textbf{0}} & \hl{\textbf{0}} & -4 & 4   \\
  \hl{\textbf{0}} & \hl{\textbf{0}} & \hl{\textbf{0}} & \hl{\textbf{24}}  \\
\end{pmatrix}
$

\vspace{5mm} %5mm vertical space

1*(-1)*(-4)*24=\hl{\textbf{96}} \\

\vspace{4mm} %5mm vertical space

S= det(A) = \hl{\textbf{96}} \\

\newpage

H) Calcul du déterminant

\vspace{5mm} %5mm vertical space

$
A =
\begin{pmatrix}
  1 & 5 & 6 & 7 \\
  0 & 2^0-1 & 1-2^32^{-3} & 8 \\
  9 & 9,5 & -9,5 & b \\
  4 & 8 & 16 & 32 \\
\end{pmatrix}
$

\vspace{10mm} %5mm vertical space
Simplification de la matrice \\

\vspace{2mm} %5mm vertical space
$2^{0}-1 = 1-1 = 0$ $ $ et  $ $ $1-2^{3} 2^{-3} = 1-2^{3-3} = 1-2^{0} = 1-1 = 0$
\vspace{4mm} %5mm vertical space

$
\begin{pmatrix}
  1 & 5 & 6 & \hl{\textbf{7}} \\
  \hl{\textbf{0}} & \hl{\textbf{0}} & \hl{\textbf{0}} & \hl{\textbf{8}} \\
  9 & 9,5 & -9,5 & \hl{\textbf{b}} \\
  4 & 8 & 16 & \hl{\textbf{32}} \\
\end{pmatrix}
$

\vspace{5mm} %5mm vertical space
Extraction Matrice 3*3 \\

$
8*
\begin{pmatrix}
  1 & 5 & 6 \\
  9 & 9,5 & {-9,5} \\
  4 & 8 & 16 \\
\end{pmatrix}
$
$
\begin{pmatrix}
  + & + & - \\
  - & - & + \\
  + & + & - \\
\end{pmatrix}
$

\vspace{10mm} %5mm vertical space
Extraction des matrices 2*2\\
\vspace{3mm} %5mm vertical space

$
8*(
  1*
  \begin{pmatrix}
    9,5 & -9.5 \\
    8 & 16 \\
  \end{pmatrix}
  $
  $
  -5*
  \begin{pmatrix}
    9 & -9.5 \\
    4 & 16 \\
  \end{pmatrix}
  $
  $
  +6*
  \begin{pmatrix}
    9 & 9.5 \\
    4 & 8 \\
  \end{pmatrix}
  $
)

\vspace{10mm} %5mm vertical space
Calcul du déterminant des sous matrices \\

8*(\\

1*((9,5*16)-(8*-9,5))\\

-5*((9*16)-(4*-9,5))\\

+6*((9*8)-(4*9,5))\\

)

\vspace{10mm} %5mm vertical space
Simplification des calculs
\vspace{5mm} %5mm vertical space

8*(\\

  1*(152 – (-76))\\

  -5*(144 – (-38))\\

  +6*(72 – 38)\\

)

\vspace{20mm} %5mm vertical space
Mise en équation et résolution
\vspace{5mm} %5mm vertical space

8*( 228 -5*(182) + 6*(34)) \\

8*( 228 - 910 + 204 ) \\

8*( 228 + 204 - 910 ) \\

8*( 432 - 910 ) \\

8*( -478 ) = -3824 \\

det(A) = -3824 \\

\newpage

I) Calcul du déterminant \\

a=3 b=10 c=5 \\

$
\begin{pmatrix}
  a & 1337 \\
  b & 42 \\
  c & 8086 \\
\end{pmatrix}
$
\vspace{5mm} %5mm vertical space
$
+
\begin{pmatrix}
  2 & 5 & 6 \\
  4 & 0 & 4 \\
\end{pmatrix}
$
\vspace{5mm} %5mm vertical space

Etape 1 :  Calculer la multiplication \\

\vspace{3mm} %5mm vertical space
$
\begin{pmatrix}
  a*2+1337*4 & a*5+1337*0  & a*6+1337*4 \\
  b*2+42*4   & b*5+42*0    & b*6+42*4   \\
  c*2+8086*4 & c*5+8086*0  & c*6+8086*4 \\
\end{pmatrix}
$\vspace{5mm} %5mm vertical space

\vspace{5mm} %5mm vertical space

Etape 2 : Remplacement des valeurs \\

\vspace{5mm} %5mm vertical space
$
\begin{pmatrix}
  3*2+1337*4 & 3*5  & 3*6+1337*4  \\
  10*2+42*4  & 10*5 & 10*6+42*4   \\
  5*2+8086*4 &  5*5  & 5*6+8086*4 \\
\end{pmatrix}
$

\vspace{5mm} %5mm vertical space

$
\begin{pmatrix}
  5354 & 15 & 5366   \\
  188 &  50 & 228    \\
  32354 & 25 & 32374 \\
\end{pmatrix}
$
\vspace{5mm} %5mm vertical space
$
\begin{pmatrix}
  + & - & + \\
  - & + & - \\
  + & - & + \\
\end{pmatrix}
$
\vspace{5mm} %5mm vertical space

Etape 4 :  Extraction des matrices 2*2 \\

\vspace{5mm} %5mm vertical space

$
+5354 *(
\begin{pmatrix}
  50 & 228  \\
  25 & 32374  \\
\end{pmatrix}
)
$
$
-15 *(
\begin{pmatrix}
  188 & 228  \\
  32354 & 32374  \\
\end{pmatrix}
)
$
$
+5366 *(
\begin{pmatrix}
  188 & 50  \\
  32354 & 25  \\
\end{pmatrix}
)
$

\vspace{5mm} %5mm vertical space

+5354 *((50*32374) - (228*25)) \\

-15 *((188*32374) - (32354*228)) \\

+5366 *((188*25) - (32354*50)) \\

\vspace{5mm} %5mm vertical space

+5354 *((1618700) - (5700)) \\

-15 *((6086312) - (7376712)) \\

+5366 *((4700) - (1617700)) \\

\vspace{5mm} %5mm vertical space

8 636 002 000 + 19 356 000 - 8 655 358 000 \\

8 655 358 000 - 8 655 358 000 = 0 \\

\newpage

J) Calcul du déterminant \\

a=3 b=10 c=5 \\

$
\begin{pmatrix}
  1 & 5 & 6 & 7 \\
  0 & 2^{0}-1 & 2^{3}2^{-3} & 8 \\
  9 & 9,5 & -9,5 & b \\
  4 & 8 & 16 & 32 \\
\end{pmatrix}
$
\vspace{5mm} %5mm vertical space
$
-
\begin{pmatrix}
  a & 40 & 0 & 1 \\
  b & 80 & 1 & 2 \\
  c & 62 & 2 & 0 \\
  d & 0  & 1 & 2 \\
\end{pmatrix}
$
\vspace{5mm} %5mm vertical space

Etape 1 :  Calculer l'opération \\

\vspace{3mm} %5mm vertical space
$
\begin{pmatrix}
  1 & 5 & 6 & 7 \\
  0 & 2^{0}-1 & 2^{3}2^{-3} & 8 \\
  9 & 9,5 & -9,5 & b \\
  4 & 8 & 16 & 32 \\
\end{pmatrix}
$
\vspace{5mm} %5mm vertical space
$
+(-1)*
\begin{pmatrix}
  a & 40 & 0 & 1 \\
  b & 80 & 1 & 2 \\
  c & 62 & 2 & 0 \\
  d & 0  & 1 & 2 \\
\end{pmatrix}
$

\vspace{5mm} %5mm vertical space

$
\begin{pmatrix}
  1 & 5 & 6 & 7 \\
  0 & 2^{0}-1 & 2^{3}2^{-3} & 8 \\
  9 & 9,5 & -9,5 & b \\
  4 & 8 & 16 & 32 \\
\end{pmatrix}
$
\vspace{5mm} %5mm vertical space
$
+
\begin{pmatrix}
  -a & -40 & 0 & -1 \\
  -b & -80 & -1 & -2 \\
  -c & -62 & -2 & 0 \\
  -d & 0  & -1 & -2 \\
\end{pmatrix}
$

Etape 2 : Réalisation de l'opération \\

\vspace{5mm} %5mm vertical space

$
\begin{pmatrix}
  1 & 5 & 6 & 7 \\
  0 & 2^{0}-1 & 2^{3}2^{-3} & 8 \\
  9 & 9,5 & -9,5 & b \\
  4 & 8 & 16 & 32 \\
\end{pmatrix}
$
\vspace{5mm} %5mm vertical space
$
+
\begin{pmatrix}
  1-a & 5-40 & 6 & 7-1 \\
  -b & -80 & -1 & 8-2 \\
  9-c & 9.5-62 & -9.5-2 & b \\
  4-d & 8  & 16-1 & 32-2 \\
\end{pmatrix}
$

\vspace{5mm} %5mm vertical space

$
\begin{pmatrix}
  -2  & -35   & 6     & 6 \\
  -10 & -80   & -1    & 6 \\
  4   & -52.5 & -11.5 & 10 \\
  2   & 8     & 15    & 30 \\
\end{pmatrix}
$

\vspace{8mm} %5mm vertical space

Etape 3 : Méthodes du pivot de Gauss \\

L2 = L2-(-5)*L1=(0 -255 -31 36) \\
L3 = L3-(-2)*L1=(0 -122.5 0.5 22) \\
L4 = L4-(-1)*L1=(0 43 9 24) \\

\vspace{3mm} %5mm vertical space
$
\begin{pmatrix}
 -2 & -35    & 6     & 6 \\
  0 & -255   & 29    & 36 \\
  0 & -122.5 & 0.5 & 22 \\
  0 & 43     & 9    & 24 \\
\end{pmatrix}
$
$
\begin{pmatrix}
 + & - & + & - \\
 - & + & - & + \\
 + & - & + & - \\
 - & + & - & + \\
\end{pmatrix}
$

\newpage

Etape 4 : Extraction des sous matrice 2*2
\vspace{5mm} %5mm vertical space

$
-2*
\begin{pmatrix}
  -255   & 29   & 36 \\
  -122.5 & 0.5  & 22 \\
  43     & 9    & 24 \\
\end{pmatrix}
$
\vspace{5mm} %5mm vertical space
$
\begin{pmatrix}
  + & - & + \\
  - & + & - \\
  + & - & + \\
\end{pmatrix}
$

\vspace{5mm} %5mm vertical space

-2*(
$
+(-255)*
\begin{pmatrix}
  0.5 & 22 \\
  9   & 24 \\
\end{pmatrix}
$
$
(-29)*
\begin{pmatrix}
  122.5 & 22 \\
  43   & 24 \\
\end{pmatrix}
$
$
(36)*
\begin{pmatrix}
  122.5 & 0.5 \\
  43   & 9 \\
\end{pmatrix}
$
)

\vspace{3mm}

-2 *( \\
-255* (12-198)\\
-29* ((-2940) - (946)) \\
+36* ((-1102.5) - 21.5) \\
) \\

47430 + 112694 - 40464= 119660


\newpage

K) Calcul du déterminant \\

a=3 b=10 c=5 \\

$
\begin{pmatrix}
  a & 2 & 0 \\
  b & 5 & 1 \\
  c & 6 & 2 \\
\end{pmatrix}
$
\vspace{5mm} %5mm vertical space
$
*
\begin{pmatrix}
  4 & 5 & 6 \\
  4 & 0 & 4 \\
  a & b & c \\
\end{pmatrix}
$
\vspace{5mm} %5mm vertical space

Etape 1 :  Calculer l'addition \\

\vspace{3mm} %5mm vertical space
$
\begin{pmatrix}
  a+4 & 2+5  & 6  \\
  b+4 & 5    & 1+4 \\
  c+a & 6+b  & 2+c \\
\end{pmatrix}
$

\vspace{5mm} %5mm vertical space

Etape 2 : Remplacement des valeurs \\

\vspace{3mm} %5mm vertical space
$
\begin{pmatrix}
  7  & 7  & 6 \\
  14 & 5  & 5 \\
  8  & 16 & 7 \\
\end{pmatrix}
$
\vspace{5mm} %5mm vertical space

Etape 3 : Calcul du déterminant \\

\vspace{3mm} %5mm vertical space
$
\begin{pmatrix}
  7  & 7  & 6 \\
  14 & 5  & 5 \\
  8  & 16 & 7 \\
\end{pmatrix}
$
$
\begin{pmatrix}
  + & - & + \\
  - & + & - \\
  + & - & + \\
\end{pmatrix}
$

\vspace{5mm} %5mm vertical space

$
+7 *(
\begin{pmatrix}
  5 & 5   \\
  16 & 7  \\
\end{pmatrix}
)
$
$
-7 *(
\begin{pmatrix}
  14 & 5  \\
  8 & 7   \\
\end{pmatrix}
)
$
$
+6 *(
\begin{pmatrix}
  14 & 5  \\
  8 & 16  \\
\end{pmatrix}
)
$

\vspace{5mm} %5mm vertical space

+7 *((5*7) - (16*5)) \\

-7 *((14*7) - (8*5)) \\

+6 *((14*16) - (8*5)) \\


\vspace{5mm} %5mm vertical space

+7 *(-45) -15 *(58) + 6 *(184) \\

1104-406-315 = 383 \\
