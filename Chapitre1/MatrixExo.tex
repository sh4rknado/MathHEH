
\newpage
\section{Exercice Matrices}
\vspace{10mm} %5mm vertical space
\subsection{Enoncés des exercices}
\vspace{5mm} %5mm vertical space

$
A =
\begin{pmatrix}
  0 & 1 & 2 & 3 \\
  1 & 2 & 3 & 0 \\
  2 & 3 & 0 & 1 \\
  3 & 0 & 1 & 2 \\
\end{pmatrix}
$
\vspace{5mm} %5mm vertical space
$
B =
\begin{pmatrix}
  1 & 4 \\
  2 & 3 \\
  3 & 2 \\
  4 & 1 \\
\end{pmatrix}
$
\vspace{5mm} %5mm vertical space
$
C =
\begin{pmatrix}
  1 & 2 & 3 & 4 \\
  4 & 3 & 2 & 1 \\
\end{pmatrix}
$
\vspace{3mm} %5mm vertical space

\begin{enumerate}[label=\Alph*)]
\item Calculer B*C
\item Calculer la trace de A
\item Calculer la transposée de B
\item Calculer 2,5*C
\item Calculer $B^{t}+C$
\item Calculer le déterminants de A
\item Exercices d'examens
\item Exercices supplémentaire (Déplacement 3D)
\end{enumerate}

\newpage

\subsection{Résolution des exercices}
\vspace{5mm} %5mm vertical space
A) Calculer B*C
\vspace{10mm} %5mm vertical space

$
B*C =
\begin{pmatrix}
  1*1+4*4 & 1*2+4*3 & 1*3+4*2 & 1*4+4*1 \\
  2*1+3*4 & 2*2+3*3 & 2*3+3*2 & 2*4+3*1 \\
  3*1+2*4 & 3*2+2*3 & 3*3+2*2 & 3*4+2*1 \\
  4*1+1*4 & 4*2+1*3 & 4*3+1*2 & 4*4+1*1 \\
\end{pmatrix}
$

\vspace{5mm} %5mm vertical space

$
S = B*C =
\begin{pmatrix}
  17 & 14 & 11 & 8 \\
  14 & 13 & 12 & 10 \\
  11 & 12 & 13 & 15 \\
  8 & 11 & 14 & 17 \\
\end{pmatrix}
$

\vspace{10mm} %5mm vertical space
B) Calculer la trace de A
\vspace{5mm} %5mm vertical space

La trace d'une matrices est la somme de chaque éléments de sa diagonale
\vspace{5mm} %5mm vertical space

$
A =
\begin{pmatrix}
  \hl{\textbf{0}}  & 1 & 2 & 3 \\
  1 & \hl{\textbf{2}} & 3 & 0 \\
  2 & 3 & \hl{\textbf{0}} & 1 \\
  3 & 0 & 1 & \hl{\textbf{2}} \\
\end{pmatrix}
$
\vspace{3mm} %5mm vertical space

La trace de la matrice A = 0+2+0+2 = 4 \\
S = 4 \\

\vspace{10mm} %5mm vertical space
C) Calculer la transposée de la matrice B

\vspace{4mm} %5mm vertical space
La transposée de la matrice est d'intervertir les lignes/colonnes de la matrice originale.

\vspace{5mm} %5mm vertical space
$
B =
\begin{pmatrix}
  1 & 4 \\
  2 & 3 \\
  3 & 2 \\
  4 & 1 \\
\end{pmatrix}
$
\vspace{5mm} %5mm vertical space
$
B^{t} =
\begin{pmatrix}
  1 & 2 & 3 & 4 \\
  4 & 3 & 2 & 1 \\
\end{pmatrix}
$

Notes : $B^{t}$ est égale à C \\

$
B^{t} = C =
\begin{pmatrix}
  1 & 2 & 3 & 4 \\
  4 & 3 & 2 & 1 \\
\end{pmatrix}
$

\vspace{5mm} %5mm vertical space

S = $B^{t}$ ou C\\

\vspace{10mm} %5mm vertical space
D) Calculer 2,5*C

\vspace{5mm} %5mm vertical space
$
2,5*C =
\begin{pmatrix}
  1*2,5 & 2*2,5 & 3*2,5 & 4*2,5 \\
  4*2,5 & 3*2,5 & 2*2,5 & 1*2,5 \\
\end{pmatrix}
$
\vspace{5mm} %5mm vertical space

$
S = 2,5*C =
\begin{pmatrix}
  2,5 & 5 & 7,5 & 10 \\
  10 & 7,5 & 5 & 2,5 \\
\end{pmatrix}
$


\vspace{10mm} %5mm vertical space
E) Calculer $B^{t} + C$

\vspace{5mm} %5mm vertical space
$
B^{t} = C =
\begin{pmatrix}
  1 & 2 & 3 & 4 \\
  4 & 3 & 2 & 1 \\
\end{pmatrix}
$
\vspace{5mm} %5mm vertical space

Notes : $B^{t} = C$ = C+C ou 2*C
\vspace{5mm} %5mm vertical space

$
S = 2*C =
\begin{pmatrix}
  1*2 & 2*2 & 3*2 & 4*2 \\
  4*2 & 3*2 & 2*2 & 1*2 \\
\end{pmatrix}
$

\vspace{5mm} %5mm vertical space
S = $ B^{t}$ +C = 2*C = C+C = \\

$
\begin{pmatrix}
  2 & 4 & 6 & 8 \\
  8 & 6 & 4 & 2 \\
\end{pmatrix}
$
