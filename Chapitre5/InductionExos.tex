\newpage

\chapter{Preuve par induction Exercices}
\vspace{5mm} %5mm vertical space

\section{Prouver par induction}

\[
  \sum_{k=0}^{n} (2k + 1)= (n+1)^{2}, dans N
\]

\textbf{Cas de base : n = 0} \\

1er parties : \\

(2k + 1) où k = 0 \\
(2*0 + 1) = 1 \\

2e parties : \\

$(n+1)^{2}$ où n = 0 \\
$(0+1)^{2}$ = 1 \\

\textbf{Conclusion : la première partie $((2*0 + 1) = 1)$ == la deuxième parties $((0+1)^{2} = 1)$} \\

\textbf{Cas général : on suppose que pour n c'est vérifié, on va essayer pour n+1 } \\

1e parties : \\

\[
  \sum_{k=0}^{n+1} (2k + 1) =
  \sum_{k=0}^{n} (2k + 1) + (2k+1)
\]

\[
  \sum_{k=0}^{n+1} (2k + 1) = (n+1)^{2} + (2(n+1)+1)
\]

\[
  \sum_{k=0}^{n+1} (2k + 1) = n^{2} + 2*n*1 +1^{2} + 2n + 2 + 1
\]

\[
  \sum_{k=0}^{n+1} (2k + 1) = n^{2} + 4n + 4
\]

2e parties : \\

$((n+1)+1)^{2}$ \\
$(n+2)^{2}$ \\
$(n+2)^{2}$ \\
$n^{2}+4n+4$ \\


\textbf{Conclusion : la première partie $(n^{2}+4n+4)$ == la deuxième parties $(n^{2}+4n+4)$} \\

\newpage

\textbf{2) Prouvez par induction} \\

\[
  \sum_{k=0}^{n} 2^{k} = 2^{n+1}-1, dans N
\]

\textbf{Cas de base : n = 0} \\

1er parties : \\

$2^{n+1}-1$ \\
$2^{1}-1$ = 1 \\

2e parties : \\

$2^{0}$ = 1 \\

\textbf{Conclusion : la première partie ($2^{0}$=1) == la deuxième parties ($2^{1}-1$ = 1)} \\

\vspace{6mm}

\textbf{Cas général : on suppose que pour n c'est vérifié, on va essayer pour n+1 } \\

1e parties : \\

\[
  \sum_{k=0}^{n+1} 2^{k} =
  \sum_{k=0}^{n} 2^{k} + 2^{n+1}
\]

\[
  \sum_{k=0}^{n+1} 2^{k} = 2^{n+1}-1 + 2^{n+1}
\]

\[
  \sum_{k=0}^{n+1} 2^{k} = 2^{n+2}-1
\]


2e parties : \\

$2^{n+1+1}-1$ \\
$2^{n+2}-1$ \\

\textbf{Conclusion : la première partie $(2^{n+2}-1)$ == la deuxième parties $(2^{n+2}-1)$} \\

\newpage
