\newpage

\section{Relation Binaire Exercices}
\vspace{5mm} %5mm vertical space

\begin{itemize}
\item {a. $R=\{(a,b), a ∈ N, b ∈ N | $ a est un multiple de b \} }
\item {b. $R=\{(a,b), a ∈ N, b ∈ N |$ a est $>$ b \} }
\item {c. $R=\{(a,b), a ∈ N, b ∈ N |$ b est divisible a \} }
\end{itemize}

\subsection{Exercices Examen}
\vspace{3mm} %5mm vertical space

Soit N est l’esemble des naturels sauf 0 \\
$R=\{(a,b), a ∈ N, b ∈ N | $ a est un multiple de b \} \\

\vspace{3mm} %5mm vertical space
cochez ce qui est vrai concernant R : \\

\begin{itemize}[label=$\square$]
\item {a. R est transitif}
\item {b. Aucune réponse}
\item {c. R est réflexif}
\item {d. R est anti-symètrique}
\item {e. R est symètrique}
\end{itemize}

\vspace{5mm} %5mm vertical space

Test de la Réflexivité \\

a multiple de a = VRAI \\
b multiple de b = VRAI \\
R est réflexif \\

Test de la symétrie $=>$ Exemple (a=2 ou b=6) \\

a multiple de b = VRAI \\
b multiple de a = FAUX \\
R est n'est pas symétrique \\


Test de Anti-symétrie $=>$ (a=b) Exemple (a=3 ou b=3) \\

a multiple de b = VRAI \\
b multiple de a = VRAI \\
R est est anti-symétrique \\


Test de Transitivité $=>$ (a=b) Exemple (a=3 ou b=9 Z=18) \\

a multiple de b et b multiple de Z est-ce que A est multiple de Z ? \\
a est dans la table de 18 ?  $=>$ VRAI \\
R est transitif \\

\newpage

Soit N est l’esemble des naturels sauf 0 \\
$R=\{(a,b), a ∈ N, b ∈ N |$ a est $>$ b \} \\

\vspace{2mm} %5mm vertical space

cochez ce qui est vrai concernant R: \\

\begin{itemize}[label=$\square$]
\item {a. R est transitif}
\item {b. Aucune réponse}
\item {c. R est réflexif}
\item {d. R est anti-symètrique}
\item {e. R est symètrique}
\end{itemize}

\vspace{4mm} %5mm vertical space

Test de la réfléxivité \\

A est plus grand que A $=>$ FAUX \\
B est plus grand que B $=>$ FAUX \\
il faut que A et B soit vrai \\

Test de la symétrie \\

A est plus grand que B $=>$ VRAI \\
B est plus grand que A $=>$ FAUX \\
il faut que A et B soit vrai \\

Test de l'anti-symétrie \\

R n'est Symètrique pas car A=1 B=2 \\
R est anti-symétrique $a est > b$ car a $\neq$ b\\

Test de la transitivité \\

Si A est $>$ B et que B est $>$ Z est-ce que a $>$ Z ? \\
R est transitif car A est $>$ Z \\


\vspace{10mm} %5mm vertical space

Soit N est l’esemble des naturels sauf 0 \\
$R=\{(a,b), a ∈ N, b ∈ N |$ b est divisible a \} \\

cochez ce qui est vrai concernant R : \\

\begin{itemize}[label=$\square$]
\item {a. R est transitif}
\item {b. Aucune réponse}
\item {c. R est réflexif}
\item {d. R est anti-symètrique}
\item {e. R est symètrique}
\end{itemize}

Test de la réfléxivité \\

A est divisible par A $=>$ VRAI \\
B est divisible par B $=>$ VRAI \\
R est réflexif \\

Test de la symétrie \\

A est divisible par B $=>$ VRAI \\
B est divisible par A $=>$ FAUX \\
il faut que A et B soit vrai \\
R n'est pas symétrique \\

Test de l'anti-symétrie \\

R n'est Symètrique pas car A=1 B=2 \\
R est anti-symétrique $a est > b$ car a $\neq$ b\\

Test de la transitivité \\

Si A est divisible par B et que B est divisible par Z est-ce que a divisible par Z ? \\
R est transitif car A est divisible par Z \\
