\newpage

\chapter{Chaptire 3 : Logique}

\section{Logique propositionnelle}
\vspace{5mm} %5mm vertical space

Règles pour déterminer si c'est vrai ou faux: \\

1) Principe d'identité : A=A \\

2) Non contradiction : On ne peut pas nier et affirmer la même chose ¬A et A \\

3) Tiers Exlus : Quelques chose existe ou dois ne pas exister A ou ¬A \\


\subsection{Proposition}
\vspace{5mm} %5mm vertical space

C'est un énoncé, une phrase simple : \\
ex: Ceci est une vidéos $=>$ Vrai ou Faux \\

En logique propositionnelle les propositions ne peuvent qu'être vrai ou fausse \\

exemple de proposition : \\
2+2 $=>$ Vrai ou Faux \\
Le mur est blanc $=>$ Vrai ou Faux \\

\subsection{L'implication}
\vspace{5mm} %5mm vertical space

Si j'ai une proposition A alors B \\

Exemple: \\
Une paire de chaussure $=>$ j'ai 2 chaussures \\
Une paire de chaussure implique que j'ai 2 chaussures \\
A$=>$B : Faux (une paire nécessite d'avoir 2 même chaussures, 2 chaussures peuvent être différentes)\\
Si A est vrai alors B est vrai \\
si B est vrai alors A n'est pas forcément vrai \\

\subsection{L'équivalence}
\vspace{5mm} %5mm vertical space

Il faut que je n'ai pas une paires de chaussures. \\
A=B : vrai \\
Si A est vrai alors B est vrai \\
si B est vrai alors A est vrai \\

\subsection{Vocabulaire}
\vspace{5mm} %5mm vertical space

Proposition Atomique : Vrai et Faux à la fois \\
Tautologie : toujours vrai \\
prédicats :  Pour tout il existe \\



\subsection{Tableau priorités logique}
\begin{tabular}{|l|c|c|c|c|}
  \hline
  Opérateur & Logic & priorités & Associativités .\\
  \hline
  $<=>$ & Equalité & 1 & gauche \\
  $=>$ & Implications & 2 & droite \\
  V & OU & 3 & gauche \\
  ∧ & ET & 4 & gauche \\
  ¬ & NON & 5 & gauche \\
  \hline
\end{tabular}


\subsection{Tautologie}
\begin{tabular}{|l|c|c|c|}
  \hline
  P & ¬ P & P V ¬P \\
  \hline
  T & T & ⊥ \\
  ⊥ & T & T \\
  \hline
\end{tabular}

\subsection{Changement de forme}
\vspace{5mm} %5mm vertical space
Commutativité: \\
pvq = qvp \\
p∧q = q∧p \\

\vspace{5mm} %5mm vertical space
Associativités: \\
(pvq)vr = pv(qvr) \\
(p∧q)∧r = p∧(q∧r) \\

\vspace{5mm} %5mm vertical space
Distributivités: \\
pv(q∧r) = (pvq)∧(pvr) \\
pv(qvr) = (p∧q)v(p∧r) \\

\vspace{5mm} %5mm vertical space
De Morgans: \\
a v b= ¬a * ¬b\\
a*b= ¬a + ¬b\\
(p∧q) = ¬p v ¬q  \\
(pvq) = ¬ (¬p ∧ ¬q)  \\
¬(p∧q) = (p v q)  \\
(A ∧¬ B) V (¬ A V (C ∧ A)) =  ¬(A ∧¬ B) ∧ ¬(¬ A V (C ∧ A))\\

\vspace{5mm} %5mm vertical space
Forme disjonctive: \\
(A ∧ B) V C\\
(A ET B) OU C \\

\vspace{5mm} %5mm vertical space
Forme conjonctive : \\
(A V B) ∧ C\\
(A OU B) ET C \\

\vspace{5mm} %5mm vertical space
Transformation: \\
A$=>$B = ¬A v (A∧B) \\
A$<=>$B: = (A$=>$B)∧(B$=>$A)  \\
(A$=>$B)∧(B$=>$A) = (¬A v (A∧B)) ∧ (¬B v (B∧A))
