\newpage

\chapter{Logique propositionnelle exercices}
\vspace{5mm} %5mm vertical space

\section{Enoncé Exercices}

P1=(1+1=2) $ $ P2=(1$>$5) $ $ P3=(1+1=3) \\


\textbf{1) Déterminer la véracité} \\

a. $P_1$ v $P_3$ \\
b. $P_2 => P_1$ \\
c. $P_3 => (P_1$ v $P_3)$ \\

\vspace{3mm}
\textbf{2) Construire la Table de vérité de $P_1 <=> P_2 => P_3$} \\

\vspace{3mm}
\textbf{3) Prouver l'équivalence} \\

a. P V P $<=>$ P \\
b. P ∧ P  $<=>$ P \\
c. P V (q ∧ r) $<=>$ (p v r) ∧ (q v r) \\

\vspace{3mm}
\textbf{4) Déterminer la véracité} \\

Soit A = \{1, 2, 3, 4, 5\} le domaine des prédicats (liste exhaustive des prédicats). Vérifiez la véracité de \\

a) $∃x$, $(x + 3 = 5)$ \\
b) $∃x$, $(x + 1 = 15)$ \\
c) $∀x$, $(x < 4)$ \\
d) $∀x$, $(x + 10 < 25)$ \\
e) $∀x$, $((x > 6) => (x < 2))$ \\
f) $∃x$, $((x^{2} = 121) ∧ (x > 0))$ \\

Soient x i ∈ A, P($x_{i}, x_{j}$) (toutes valeurs possibles), vérifiez les tautologies

\newpage
\section{Déterminer la véracité}

\begin{tabular}{|l|c|c|c|c|c|c|c|}
  \hline
  $P_1$ & $P_2$ & $P_3$ & $P_1$ v $P_3$ & $P_2 => P_1$ & ($P_1$ v $P_3$) & $P_3 => (P_1$ v $P_3)$ \\
  \hline
  T & ⊥ & ⊥ & T & T & T & T \\
  \hline
\end{tabular}


\section{Construire la table de vérité}
\vspace{5mm} %5mm vertical space

$P_1 <=> P_2 => P_3$ \\

\begin{tabular}{|l|c|c|c|c|c|}
  \hline
  $P_1$ & $P_2$ & $P_3$ & $P_2 => P_3$ & $P_1 <=> (P_2 => P_3)$ \\
  \hline
  0 & 0 & 1 & 1 & 0 \\
  \hline
  0 & 1 & 0 & 0 & 1 \\
  \hline
  0 & 1 & 1 & 1 & 0 \\
  \hline
  1 & 0 & 0 & 1 & 1 \\
  \hline
  1 & 0 & 1 & 1 & 1 \\
  \hline
  1 & 1 & 0 & 0 & 0 \\
  \hline
  1 & 1 & 1 & 1 & 1 \\
  \hline
\end{tabular}

\section{Prouver l'équivalence}

\vspace{3mm} %5mm vertical space
a. P V P $<=>$ P \\

\begin{tabular}{|l|c|c|c|c|}
  \hline
  P & P V P & P $<=>$ P V P \\
  \hline
  0 & 0 & 0 \\
  \hline
  1 & 1 & 1 \\
  \hline
\end{tabular}

\vspace{3mm}
b. P ∧ P  $<=>$ P \\

\begin{tabular}{|l|c|c|c|c|}
  \hline
  P & P ∧ P & P $<=>$ P ∧ P \\
  \hline
  0 & 0 & 0 \\
  \hline
  1 & 1 & 1 \\
  \hline
\end{tabular}

\vspace{5mm}
c. P v (q ∧ r) $<=>$ (p v r) ∧ (q v r) \\

utilisation de la distributivités \\

P v (q ∧ r) $<=>$ (p v r) ∧ (q v r) \\

(p v r) ∧ (q v r) $<=>$ (p v r) ∧ (q v r) \\

\newpage

\section{Déterminer la véracité}

a) $∃x$, $(x + 3 = 5)$ \\

x+3=5 \\
x=5-3 \\
x=2 \\

x=2, il existe bien un x appartenant à A \\

\vspace{5mm}

b) $∃x$, $(x + 1 = 15)$ \\

x+1=15 \\
x=14 \\

Il n'existe pas de x compris dans A, car 14 n'est pas compris dans A \\

\vspace{5mm}

c) $∀x$, $(x < 4)$ \\

\begin{tabular}{|l|c|c|c|c|}
  \hline
  $1 < 4$ & T \\
  \hline
  $2 < 4$ & T \\
  \hline
  $3 < 4$ & T \\
  \hline
  $4 < 4$ & ⊥ \\
  \hline
  $5 < 4$ & ⊥ \\
  \hline
\end{tabular}

\vspace{4mm}

x=4 ou x=5 n'est pas inférieure à 4, donc tout x n'est pas inférieure à 4 \\

\vspace{5mm}

d) $∀x$, $(x + 10 < 25)$ \\

\begin{tabular}{|l|c|c|c|c|}
  \hline
  $(1 + 10 < 25)$ & T \\
  \hline
  $(2 + 10 < 25)$ & T \\
  \hline
  $(3 + 10 < 25)$ & T \\
  \hline
  $(4 + 10 < 25)$ & T \\
  \hline
  $(5 + 10 < 25)$ & T \\
  \hline
\end{tabular}

\vspace{4mm}

x+10 = 25 \\
x = 15 \\

Tout x appartenant à A, ou x+10 est toujours inférieure a 25. \\

De plus X $>=$ 15 n'appartient pas à A. \\

\newpage

e) $∀x$, $((x > 6) => (x < 2))$ \\

\begin{tabular}{|l|c|c|c|c|}
  \hline
  x & $A=(x > 6)$ & $B=(x < 2)$ & $A => B$ \\
  \hline
  1 & ⊥ & ⊥ & T \\
  \hline
  2 & ⊥ & ⊥ & T \\
  \hline
  3 & ⊥ & ⊥ & T \\
  \hline
  4 & ⊥ & ⊥ & T \\
  \hline
  5 & ⊥ & ⊥ & T \\
  \hline
\end{tabular}

\vspace{4mm}

Quand quelques chose de chose de faux est impliqué, le résultat est toujours vrai. \\

⊥ $=>$ T $<=>$ T \\
⊥ $=>$ ⊥ $<=>$ T \\

\begin{tabular}{|l|c|c|c|}
  \hline
  P & Q & P $=>$ Q \\
  \hline
  T & T & T \\
  T & ⊥ & ⊥ \\
  ⊥ & T & T \\
  ⊥ & ⊥ & T \\
  \hline
\end{tabular}

\vspace{10mm}

f) $∃x$, $((x^{2} = 121) ∧ (x > 0))$ \\

$x^{2}$ = 121 \\
x = 11 \\

x n'appartient pas a A, $x> ∀x$ appartenant à A \\

$∀x$ appartenant à A $>$ 0 \\

 ⊥ ∧ T = T \\

\begin{tabular}{|l|c|c|c|c|}
  \hline
  x & $A=(x^{2} = 121)$ & $B=(x > 0)$ & $A ∧ B$ \\
  \hline
  1 & ⊥ & T & T \\
  \hline
  2 & ⊥ & T & T \\
  \hline
  3 & ⊥ & T & T \\
  \hline
  4 & ⊥ & T & T \\
  \hline
  5 & ⊥ & T & T \\
  \hline
\end{tabular}

\vspace{4mm}

2) Soient x i ∈ A, P(x$_{i}$ , x$_{j}$) (toutes valeurs possibles), vérifiez les tautologies suivantes : \\

a) ∀x$_{1}$ ∃x$_{2}$ P(x$_{1}$, x$_{2}$) $<=>$ ∃x$_{2}$ ∀x$_{1}$ P(x$_{1}$, x$_{2}$) \textbf{Faux} \\

b) ∃x$_{1}$ , ∀x$_{2}$ P(x$_{1}$, x$_{2}$) $=>$ ∀x$_{2}$ ∃x$_{1}$ P(x$_{1}$, x$_{2}$) \textbf{Vrai} \\

c) ∀x, P(x) $=>$ ∃x, P(x) \textbf{Vrai} \\

\newpage

Lorsque les prédicats sont dans N, Z, Q et R, vérifiez : \\

a) ∀x, ∃y, (y $<$ x) \\

N = -1 < 0 et -1 $\notin$ N : \textbf{Faux} \\
Z = -1 < 0 et -1 $\in$ Z : \textbf{Vrai} \\
Q = 1/2 < 1 et 1/2 $\in$ Q : \textbf{Vrai} \\
R = 0.60 < 1 et 0.60 $\in$ R : \textbf{Vrai} \\

b) ∀x$_{1}$, ∀x$_{2}$ (($x_{1}$ $<$ x $x_{2}$ ) $=>$ ∃y ($x_{1}$ $<$ y $<$ $x_{2}$ )) \\

N = 0 < 1 $=>$ 0 $<$ y $<$ 1 : \textbf{Faux} \\
Z = 0 < 1 $=>$ 0 $<$ y $<$ 1 : \textbf{Faux} \\
Q = 0.11 < 0.12 $=>$ 0.11 $<$ 0.115 $<$ 0.12 : \textbf{Vrai} \\
R \textbf{Vrai} \\

c)∃x, ($x_{2}$ = 2) \\

x=$\sqrt{2}$ $-\sqrt{2}$

N = $\sqrt{2}$ $\notin$ N : \textbf{Faux} \\
Z = $\sqrt{2}$ $\notin$ Z : \textbf{Faux} \\
Q = $\sqrt{2}$ $\notin$ Q : \textbf{Faux} \\
R = $\sqrt{2}$ $\in$ R: \textbf{Vrai} \\

d)∃x, ($x^{2}$ + 1 = 0) \\

∃x $=>$ $x^{2}$ = -1) \\

N = $x^{2}$ = -1 $\notin$ N : \textbf{Faux} \\
Z = $x^{2}$ = -1 $\notin$ Z : \textbf{Faux} \\
Q = $x^{2}$ = -1 $\notin$ Q : \textbf{Faux} \\
R = $x^{2}$ = -1 $\notin$ R: \textbf{Faux} \\

Soient P(x) = x est un multiple de 2 et Q(x) = x est un multiple de 4, les deux prédicats de domaine N \\

P(x) = x est un multiple de 2 \\
Q(x) = x est un multiple de 4 \\

a) ∀x,(P(x) ⇒ Q(x)) \\

Tout x multiple de 2 est un multiple de 4, si x=2, x est multiple de 2 mais pas de 4 : \textbf{Faux} \\

b) ∀x,(Q(x) ⇒ P(x)) \\

Tout x multiple de 4 est un multiple de 2 : \textbf{Vrai} \\

\newpage
c) ∃x,(P(x) ⇒ Q(x)) \\

Il existe au moins 1x qui est multiple de 2 et de 4 Vrai car si x=4, x est multiple de 2 et de 4 \\

d) ¬(∀x (P(x) ⇒ ¬Q(x))) \\

Il faut montrer que ∀x (P(x) $=>$ ¬Q(x)) est faux : \\

il faut ∃x, (P(x) ⇒ ¬¬Q(x)) = ∃x (P(x) $<=>$ Q(x )) \\

∃x (P(x) $<=>$ Q(x)) : Il existe au moins 1x qui est multiple de 2 et de 4 Vrai car si x=4, x est multiple de 2 et de 4 \\

\vspace{6mm}

Soit R le domaine des prédicats. Trouvez des formules de la logique des prédicats ne faisant pas apparaître de quantificateurs (∀,∃) équivalentes aux formules suivantes : \\

a) ∃x, ax + b = 0 \\

ax = - b \\
x = $\frac{-b}{a}$ \\

b) (a $\neq$ 0) ∧ (∃x, ax$^{2}$ + bx + c = 0) \\

(a $\neq$ 0) V (b$^{2}$ -4ac ≥ 0) où a, b et c sont dans R
