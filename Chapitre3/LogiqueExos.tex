\newpage

\chapter{Logique propositionnelle exercices}
\vspace{5mm} %5mm vertical space

\section{Enoncé Exercices}
\vspace{5mm} %5mm vertical space

1) Déterminer la véracité \\

P1 = 1+1=2  \\
P2 = 1$>$5 \\
P3 = 1+1=3 \\

\begin{itemize}
\item {a. $P_1$ v $P_3$}
\item {b. $P_2 => P_1$}
\item {c. $P_3 => (p_1$ v $P_3)$}
\end{itemize}

2) Construire la Table de vérité de $p_1 <=> P_2 => P_3$ \\

\vspace{5mm} %5mm vertical space
\section{Déterminer la véracité}

a. $P_1$ v $P_3$ = T \\
1 OU 1 = 1\\

b. $P_2 => P_1$ \\
¬ $P_2$ v ($P_2$ ∧ $P_1$) \\
¬ 0 v (0 ∧ 1) \\
1 v (0) \\
1 OU 0 = 1\\
S = $P_2 => P_1$ = T \\

c. $P_3 => (p_1 v p_3)$ \\
¬ $P_3$ v ($p_3$ ∧ ($p_1$ v $p_3$)) \\
$p_3$=0 \\
$p_1$ =1 ou insertion \\
¬ 0 v (0 ∧ (1 v 0)) \\
1 v (1 ∧ 0) \\
1 v 0 = T \\
1 OU 0 = 1 \\
S= $P_3 => (p_1 v p_3)$ = T\\

\section{Construire la table de vérité}
\vspace{5mm} %5mm vertical space
$p_1 <=> P_2 => P_3$ \\

$P_2 => P_3$ \\
¬ $P_2$ v ($p_2$ ∧ $p_3$) \\
¬ 0 v (0 ∧ 0) \\
1 v 0 = T \\
1 OU 0 = 1 \\

\begin{tabular}{|l|c|c|c|c|}
  \hline
  $P_1$ & $P_2$ & $P_3$ & $P_2 => P_3$ \\
  \hline
  T & ⊥ & ⊥ & T \\
  \hline
\end{tabular}
